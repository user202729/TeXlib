%! TEX program = lualatex
\documentclass{article}
\usepackage{prettytok}
\prettyinitterm
\usepackage{imperative}
\usepackage{l3benchmark}
\usepackage{minitest}
\usepackage{genvar}
\ExplSyntaxOn
\GenerateVariantsFile:n {test_imperative7.tex}
\ExplSyntaxOff
\directlua{imperative_debug=true}
\errorcontextlines=3
\begin{document}
\ExplSyntaxOn

\precattl_exec:n {



}

\begin{imperativerun}

\relax


% reverse #\data. Must have between 0 and 7 items.
% instead of using \ifcase / \or trick, we use this trick instead
\cs_new_eq:NN \__reverse_small \rfunction #\data{
	\putnext{#\data \exp_end: \empty \empty \empty \empty \empty \empty \empty \relax}
	\matchrm{#1 #2 #3 #4 #5 #6 #7 #8 #\remaining_empty_tokens \relax}
	\putnext{#8 #7 #6 #5 #4 #3 #2 #1}
	\passcontrol  % following in the token list is \empty \empty ... \empty \exp_end: followed by the desired resulting token list
}

% usage: 
%   \__collect_reverse_i {... result ...} ...8 more tokens...
% r-expands to
%   \__collect_reverse_i {⟨8 more tokens, reversed⟩ ... result ...}
\cs_new:Npn \__collect_reverse_i #1 #2 #3 #4 #5 #6 #7 #8 #9 {
	\exp_end: {#9 #8 #7 #6 #5 #4 #3 #2 #1}
}

\zfunction{

	% ======== define \__collect_reverse_ii, \__collect_reverse_iii etc.
	% which collects 16, 32, 64 etc. tokens respectively.
	\assign #\prev{1}
	\forint #x{2}{20}{

		\backquote{
			\ucalllocal{
				\cs_new:Npn ,c{__collect_reverse_\romannumeral#x} {
					\expandafter ,c{__collect_reverse_\romannumeral#\prev} \exp:w ,c{__collect_reverse_\romannumeral#\prev}
				}
			}
		}

		\assignnumber #\prev {#x}
	}

	%% ======== test...
	%\assign #\result {}
	%\forint #x{0}{200}{
	%	\backquote{
	%		\assign #\result {#\result ,c{#x}}
	%	}
	%}

	%\putnext{{} #\result}
	%\rcall{\__collect_reverse_iv}
	%\prettyw
}\prettystop

% then define \__throw_next which
%   \__throw_next {i} A } B
% r-expands to
%   A } i B
% admittedly brace hack is not necessary here, but it's more convenient that way
\cs_new_eq:NN \__throw_next \rfunction{
	\putnextbgroup
	\matchrm{#\block}  % block = {#i} A
	\putnext{#\block \relax}
	% input stream state: {#i} A \relax B
	\matchrm{ #i #A \relax }
	\putnext{ \iffalse{\fi  #A} #i }
	\expandonce
}


% finally, the reverse function
\cs_new_eq:NN \__reverse \rfunction #1 {
	\assignnumber #\count {\str_count:n {#1}}
	\putnext{ \iffalse{\fi #1 } }
	\expandonce

	% throw count: a → 2**(a+2)
	% i → 8
	% ii → 16
	% iii → 32
	% ...

	\while {}{\texconditional{\ifnum #\count>2049~}}{
		\assignnumber #\count{#\count-2049}
		\rcall{\__collect_reverse_ix}
		% if we don't give the {} it will collect-reverse one more
		\rcall{\__throw_next}
	}

	\while {}{\texconditional{\ifnum #\count>7~}}{
		\assignnumber #\count{#\count-8}
		\rcall{\__collect_reverse_i{}}
		\rcall{\__throw_next}
	}
	
	% okay done
	\putnextbgroup
	\rcall{\__reverse_small}
}


% ======== unexpandable linear implementation ========
% (apparently both \tl_range and \str_range[_ignore_spaces] work in quadratic time. Have to write my own, then)


% there's a little problem, we need edef for this to be linear, but


% because of a (presumably) TeX bug, implicit character tokens # are treated the same way as explicit
% anyway, use '# ⟨the token⟩' suffices (results in the token stored in the macro), as well as \unexpanded{⟨the token⟩}


\int_new:N \__num_block_left


% bdef (callback-based return) not defined yet. For now stick with this.
\cs_new:Npn \cC{__throw_after_level_0} #1 #2 {#2 #1}  % maybe '\unexpanded{#2} #1' to be safe, but they're all characters anyway, not # or something expandable

% hand implemented optimization
\cs_new:Npn \cC{__throw_after_level_1} #1 #2#3 {#2#3 #1}
\cs_new:Npn \cC{__throw_after_level_2} #1 #2#3#4#5 {#2#3#4#5 #1}
\cs_new:Npn \cC{__throw_after_level_3} #1 #2#3#4#5#6#7#8#9 {#2#3#4#5#6#7#8#9 #1}

\zblock{
	\assign #\last {\cC{__throw_after_level_3}}
	\forint #x {4} {20} {
		\assignc #\cur {__throw_after_level_#x}
		\ucalllocal{
			\cs_new:Npn #\cur ##1 {#\last {#\last {##1}} }
		}
		\assign #\last{#\cur}
	}
}

% such that, in x-expansion context,
% \__throw_after_level_0 {⟨callback⟩} {⟨item⟩} → ⟨item⟩ ⟨callback⟩
% \__throw_after_level_1 {⟨callback⟩} {⟨item⟩} {⟨item⟩} → ⟨item⟩ ⟨item⟩ ⟨callback⟩
% \__throw_after_level_X {⟨callback⟩} ⟨2**X optionally braced {item}s⟩ → ⟨2**X unbraced items⟩ ⟨callback⟩
%

\assertequal:xn { \cC{__throw_after_level_3} \use_none:n 12345678 9 0} {12345678 0}

% works like this, if \__num_block_left=3
%   \xdef \__tmp { \__throw_after_level_0 \__special_processor_at_level_0  123456789 }
% → mostly equivalent to...
%   \xdef \__tmp { 1 } \toks3=\expandafter{\__tmp} 
%   \xdef \__tmp { 2 } \toks2=\expandafter{\__tmp} 
%   \xdef \__tmp { 3 } \toks1=\expandafter{\__tmp} 
%   \xdef \__tmp { 456789 }
%
% where the number of items in each block is equal to 2**⟨level⟩

% \__num_block_left must be strictly positive.

\zblock{
	\forint #x {0} {20} {
		\backquote{
			\ucalllocal{
				\cs_new:Npn ,c{__special_processor_at_level_#x} {
						\iffalse{\fi
					}
					\toks\__num_block_left=\expandafter{\__tmp}
					\int_decr:N \__num_block_left
					\xdef \__tmp{
						\iffalse}\fi
						\int_compare:nNnF {\__num_block_left}={0} {,c{__throw_after_level_#x} ,c{__special_processor_at_level_#x}}
				}
			}
		}
	}
}

% let's try it
\__num_block_left=3~

\begingroup
\xdef \__tmp {\cC{__throw_after_level_0} \cC{__special_processor_at_level_0} 123456789}
\assertequal:on{\the\toks3}{1}
\assertequal:on{\the\toks2}{2}
\assertequal:on{\the\toks1}{3}
\assertequal:on{\__tmp}{456789}
\endgroup


\begingroup
\__num_block_left=3~
\xdef \__tmp {\cC{__throw_after_level_1} \cC{__special_processor_at_level_1} 123456}
\assertequal:on{\the\toks3}{12}
\assertequal:on{\the\toks2}{34}
\assertequal:on{\the\toks1}{56}
\assertequal:on{\__tmp}{}
\endgroup


\int_new:N \__block_level
\int_new:N \__input_size
\int_new:N \__num_block
\int_new:N \__index

\cs_new_eq:NN \__reverse_tok \zfunction #1 {
	\ucalllocal{
		\int_set:Nn \__input_size {\str_count:n {#1}}
		\int_set:Nn \__block_level {\fp_eval:n{floor(ln(\__input_size)/ln(4))}}  % each block would have size 2^level.
		\int_set:Nn \__num_block {\int_div_truncate:nn {\__input_size} {
				% here I'm lazy
				\fp_eval:n {round(2**\__block_level)}
		}}
		% there would be \__num_block complete blocks and some "remaining" items (possibly empty)


		\begingroup

			\int_set:Nn \__num_block_left {\__num_block}
			%\pretty:n{##1 = #1}

			%\int_step_inline:nnn {0}{\__num_block}  \scope #\tokindex{
			%	\pretty:no{before:\the\toks #\tokindex =~}{\the\toks #\tokindex}
			%}

			\xdef \__tmp {
				\int_compare:nNnF {\__num_block}={0}
				{
					\csname __special_processor_at_level_ \the\__block_level \endcsname
				}
				#1
			}
			\toks0=\expandafter{\__tmp}

			%\int_step_inline:nnn {0}{\__num_block}  \scope #\tokindex{
			%	\pretty:no{after\the\toks #\tokindex =~}{\the\toks #\tokindex}
			%}
			%\pretty:no{after: tmp=}\__tmp


			\int_step_inline:nnn {0}{\__num_block}  \scope #\tokindex{
				% ======== now we reverse \the\toks #\tokindex in linear time
				\begingroup  % we re-use the toksX content here, rely on the save stack operation to take constant time to restore per item

					%\pretty:no{========going to reverse}{\the\toks #\tokindex}

					\int_zero:N \__index
					\str_map_inline:on {\the\toks #\tokindex} \scope #\char {
						\int_incr:N \__index
						\toks \__index={#\char}  % will be reverted at the following \endgroup
					}
					\xdef \__tmp {
						\int_step_function:nnnN {\__index} {-1} {1} \zfunction #\index {\return{\the\toks #\index \cS\ }}
					}

					%\pretty:no{======== result is}{\__tmp}

				\endgroup  % is reverted here

				\toks #\tokindex=\expandafter{\__tmp}
			}


			% finally
			\xdef \__tmp{
				\int_step_function:nnN {0} {\__num_block} \zfunction #\index {\return{\the\toks #\index \cS\ }}
			}


		\endgroup
	}
}

\__reverse_tok {123456789}
\assertequal:oo {\detokenize{987654321}} {\__tmp}

\__reverse_tok {abcdefghi}
\assertequal:oo {\detokenize{ihgfedcba}} {\__tmp}


\cs_new_protected:Npn \__reverse_build #1 {
	\tl_build_begin:N \__tmp
	\str_map_inline:nn {#1} {
		\tl_build_put_left:Nn \__tmp {##1}
	}
	\tl_build_end:N \__tmp
}

\zfunction{
	\assign #s {abac}
	\assign #\level {17} 
	\forint #_ {0} {#\level} {
		\assign #s{#s #s}
	}
	\ucalllocal{\assertequal:xx{\str_count:n {#s}}{\fp_eval:n{2**(#\level+2)}}}
	%\pretty{#s}
	%\prettyo {\romannumeral\__reverse {#s}}

	\ucalllocal{\benchmark_tic:}
	\assignr #r{\__reverse{#s}}
	\ucalllocal{\benchmark_toc: \prettye:n {↑ ======== expandable performance ========}}

	\ucalllocal{\assertequal:xx{\str_count:n {#r}}{\str_count:n {#s}}}

	\ucalllocal{\benchmark_tic:}
	\ucalllocal{\__reverse_tok{#s}}
	\ucalllocal{\benchmark_toc: \prettye:n {↑ ======== unexpandable performance ========}}

	\ucalllocal{\assertequal_str:on{\__tmp}{#r}}

	\ucalllocal{\benchmark_tic:}
	\ucalllocal{\__reverse_build{#s}}
	\ucalllocal{\benchmark_toc: \prettye:n {↑ ======== build performance ========}}

	\ucalllocal{\assertequal_str:on{\__tmp}{#r}}

	\comment{
		\ucalllocal{\benchmark_tic:}
		\ucalllocal{\exp_args:Nx\use_none:n{\tl_reverse_items:n{#s}}}
		\ucalllocal{\benchmark_toc: \prettye:n {↑ ======== expl3 performance ========}}
	}

}

\end{imperativerun}
\ExplSyntaxOff
\end{document}
