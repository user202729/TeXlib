%! TEX program = lualatex
\ProvidesFile{unicode-math-input.tex} [2023/05/10 v0.0.0 ]
\RequirePackage{fvextra}
\documentclass{l3doc}
\usepackage[svgnames]{xcolor}
\EnableCrossrefs
\CodelineIndex
\fvset{breaklines=true,gobble=0,tabsize=4,frame=single,numbers=left,numbersep=3pt}
\AtBeginDocument{\DeleteShortVerb\"}
\MakeOuterQuote{"}
\usepackage{cleveref}
\usepackage{fontspec}
\setmonofont{DejaVu Sans Mono}[Scale=0.8]
\usepackage[bold-style=ISO]{unicode-math}
\tracinglostchars=3
\begin{document}
\hfuzz=1pt
\GetFileInfo{\jobname.tex}

\title{\pkg{unicode-math-input} --- Allow entering Unicode symbols in math formulas\thanks{This file
	describes version \fileversion, last revised \filedate.}
}
\author{user202729%
%\thanks{E-mail: (not set)}
}
\date{Released \filedate}

\maketitle

\begin{abstract}
Allow entering Unicode symbols in math formulas.
\end{abstract}

\section{Introduction}

This package allows entering Unicode symbols in math formulas.

\subsection{Existing packages}

There are several existing packages, but other than |unicode-math| (which also changes the output encoding)
they does not cover a lot of characters and/or does not handle several issues well.

We compare the situation with several existing packages:

\begin{itemize}
	\item \pkg{unixode}:
		\begin{itemize}
			\item defines |′| to be |\prime| which is big and not usable, it should be |^{\prime}|
				similar to |'|'s definition.
			\item defines |–| (en dash) to be nothing, which breaks the character even in text mode.
			\item does not define |×| or |±| (they're already valid in text mode in \LaTeX, but will be
				silently omitted in math mode)
			\item does not handle consecutive superscript/subscript characters.
			\item you need to manually patch the source code a bit in order to make it work with PDF\LaTeX. And even after that it will raise lots of warnings about redefining Unicode characters.
		\end{itemize}
	\item \pkg{utf8x}:
		\begin{itemize}
			\item incompatible with lots of packages.
			\item does not define $\bigoplus$ (|\bigoplus|)
			\item also does not handle consecutive superscript/subscript characters.
		\end{itemize}
\end{itemize}

See also \url{https://tex.stackexchange.com/a/628285}.

\subsection{Features}

\LaTeX's implementation of input encoding and font encoding is \emph{very} complicated, necessitated by the fact that non-Unicode \TeX\ engines handles each UTF-8 character as multiple tokens and enc\TeX\ extension is not enabled in \LaTeX.%
\footnote{Refer to \url{https://tex.stackexchange.com/a/266282/250119} for a way to force-enable enc\TeX\ extension in
\LaTeX\ if you're interested.}

There's a few other issues that we don't really need to deal with, because they are in the next layer:

\begin{itemize}
	\item \href{https://tex.stackexchange.com/q/31640/250119}{What is the use of the command \texttt{\textbackslash IeC}?}
	\item \url{https://tex.stackexchange.com/a/239575/250119}

\end{itemize}

We don't need to deal with |\IeC| as since \TeX\ Live 2019, the mechanism is no longer used and the Unicode character itself
is written to auxiliary files.
%\immediate\openout 5=a.tex
%\makeatletter
%\let \protect \@unexpandable@protect
%\immediate\write 5{a á b}

We need to get the following things correct:

\begin{itemize}
	\item |\left⟨|

		In Lua\LaTeX\ in order to implement this we need to hard code the |\Udelcode| of the character,
		so if |\langle| is redefined, the change will not follow.

		An alternative is to overwrite the definition of |\left| built-in, but this is not used.

	\item |\big⟨| (in \pkg{amsmath} package or outside)

		In PDF\LaTeX\ there's an issue of argument-grabbing (|\big| etc. is a macro so they will
		only grab the first octet of the |⟨| character), so the macro must be patched.

		Furthermore, the patching is done |\AtBeginDocument| in case |amsmath| etc. is loaded after this package.
		
		We handle |\big \Big \bigg \Bigg| and the |\bigl|, |\bigr| variants etc.

		Pass the option \hyperref[doc/function//ignore-patch-delimiter-commands]{|ignore-patch-delimiter-commands|} to disable the behavior in case of package clash.

	\item in \pkg{unicode-math}, |a`| renders as |a^{\backprime}| i.e. $a^{\backprime}$.
		We will not modify the default behavior i.e. $a\text{`}$ in this package.

	\item |\section{$1 × 2$}| (for writing to auxiliary file in table of contents) -- as mentioned above,
		since \TeX\ Live 2019 this is correct by default.
	\item Some characters such as |×| or |½| in PDF\LaTeX\ are already
		usable outside math mode, we try to not break the compatibility.
	\item The symbol should work correctly when appear at the start of an alignment entry,
		e.g., the start of an |align*| cell.
	\item |$2³⁴$| (consecutive Unicode characters for superscript/subscript, refer to \url{https://tex.stackexchange.com/q/344160/250119}.) Also need to handle |'| similarly.

	\item This packages does modify the default definition of |'| to allow |G'²| to work however.
		Pass the option \hyperref[doc/function//ignore-patch-prime]{|ignore-patch-prime|} to disable the behavior in case of package clash.

	\item The original implementation of |'| is somewhat interesting that it allows sequences such as |G'^\bgroup 123\egroup| to work,
		we will not emulate it here.

	\item Also need to handle Unicode prime symbols |′|, |″| etc.

	\item To minimize errors, we make |≢| default to |\nequiv|, but fallback to |\not\equiv|
		if the former is not available.

		We should also take care of aliases -- for example, |≰| should check |\nle| and |\nleq| before fallback to |\not\le| or |\not\leq|.

		Note that by default (or with \pkg{amsmath} or \pkg{amssymb}),
		|\not| does not smartly check the following symbol,
		however with some packages such as \pkg{unicode-math},
		\pkg{txfonts} the |\not| does do that -- in particular, it checks for the presence of control sequences named
		|\notXXX| and |\nXXX| where |XXX| stands for the original control sequence/character.

		It would be beneficial for \pkg{amssymb}
		to make |\not| smart, as for example |\not\exists| looks worse than |\nexists|,
		however the package does not touch |\not|.

	\item Similarly, |''| default to |^{\dprime}| if available, else fallback to |^{\prime\prime}|.

	\item Whenever possible, we do not make the symbols have active catcode, only change the mathcode,
		that way usage of the symbols in places such as |fancyvrb| environment is minimally affected. (see test files for an example)
	\item We try to make minimum assumptions about the internal implementation details of \LaTeX\ packages; nevertheless this is not always possible.

	\item Combining modifiers (such as |U+00305 COMBINING OVERLINE| in |a̅|, which corresponds to |\overbar|) are difficult to support
		(although with whole-file scanning + \pkg{rescansync} or Lua\TeX's |process_input_buffer| callback
		it's not impossible; an alternative is to use Lua\TeX\ callback to modify the math list
		after it's constructed, see \url{https://github.com/wspr/unicode-math/issues/555#issuecomment-1045207378}
		for an example),
		plus |unicode-math| does not support them anyway,
		so they will not be supported.

		They're difficult to support because normally the modifier appear after the character that it modifies
		but \TeX\ requires the command (e.g. |\overbar|) to appear \emph{before} the character that it modifies.

		As a special case, the 4 commands |\enclosecircle| |\enclosesquare| |\enclosediamond| and |\enclosetriangle|
		are supported (simply because the \TeX\ command can appear after the character it modifies)

	\item The fraction slash |U+2044 FRACTION SLASH|, as in |1|⁄|2| rendering $\frac{1}{2}$, is also not implemented
		because of similar difficulty as above.
	\item Symbols such as |√| or |∛| will be equivalent to |\sqrt| command (taking an argument to draw a square root)
		instead of |\surd| (the symbol itself), unlike \pkg{unicode-math}.

		While sequences such as |⁵√{67}| may feasibly be supported without breaking too many things,
		implementation is difficult and we don't see much use for it.
	\item Similarly, one might expect that $⏟$ |U+23DF BOTTOM CURLY BRACKET| get mapped to |\underbrace|,
		but the behavior of such command would be a bit unexpected (you need to
		write $⏟$|{123}_{456}| to get $\smash{\underbrace{123}_{456}}$), so this will not be the default.

	\item the Unicode character is mapped indirectly to the control sequence,
		so that when the user/some package redefines a control sequence such as |\pi|, the
		corresponding Unicode character (|π|) will also change. This will incur a small loss in efficiency however.

		(modulo the issue with |\Udelcode| mentioned above)
\end{itemize}

There are some issues however:
\begin{itemize}
	\item \textsf{\rlap{0}\kern 1.2pt\relax 0} |U+1D7D8 MATHEMATICAL DOUBLE-STRUCK DIGIT ZERO|
gets translated to |\mathbb{0}|, but this is incorrect by default
unless the blackboard bold font happens to have such a character.

(nevertheless, it's difficult to change math font in the middle of the document anyway.
Refer to \url{https://tex.stackexchange.com/q/30049/250119}.)

\item In the \pkg{unicode-math} source code there's this remark:

	\begin{quote}
	The catcode setting is to work around (strange?) behaviour in Lua\TeX\ in
which catcode 11 characters don't have italic correction for maths. We don't adjust
ascii chars, however, because certain punctuation should not have their catcodes
changed.
\end{quote}

This feature is currently unimplemented.
\end{itemize}




\section{Usage}

Simply include the package.

\begin{verbatim}
\usepackage{unicode-math-input}
\end{verbatim}

Because by default the \pkg{unicode-math} package will already allow entering Unicode symbols in math formulas, this package will raise an error if the other package is already loaded.

\section{Advanced commands and options}

\begin{function}{
		\umiMathbf,\umiMathit,\umiMathbfit,\umiMathscr,\umiMathbfscr,\umiMathfrak,\umiMathbb,\umiMathbbit,\umiMathbffrak,\umiMathsf,\umiMathsfbf,\umiMathsfit,\umiMathsfbfit,\umiMathtt
	}
	\begin{syntax}
		|\umiMathbf {...}|
		|\umiMathit {...}|
	\end{syntax}
	These functions are not to be used directly. But you can redefine them to customized behavior of bold/italic/etc. Unicode characters.

	For example you can |\renewcommand\umiMathbf[1]{\mathbf{#1}}| which is the default behavior.

	More usefully, you may want to |\renewcommand\umiMathbf{\bm}| to make entered characters such as
	$𝐚$ appear bold italic in the output (which is |unicode-math| behavior
	with |[bold-style=ISO]| package option).
\end{function}

\begin{function}{\umiFrac}
	\begin{syntax}
		|\umiFrac {1} {2}|
	\end{syntax}
	Not to be used directly, but you can redefine it such as |\renewcommand\umiFrac[2]{\tfrac{#1}{#2}}|
	(or shorter, |\let\umiFrac\tfrac|)
	to customize the appearance of Unicode characters like |½|.
\end{function}

\begin{function}{\umiDefineMathChar}
	\begin{syntax}
		|\umiDefineMathChar {α} {\alpha}|
	\end{syntax}
	Does what it says.

	Note that the Unicode character must be braced.

	This might or might not destroy the existing text-mode definition. For now,
	one way to preserve it is |\umiDefineMathChar {²} {\TextOrMath{\texttwosuperior}{^2}}|.
\end{function}

\begin{function}{\umiDefineMathDelimiter}
	\begin{syntax}
		|\umiDefineMathDelimiter {⟨} \langle|
	\end{syntax}
	You must use this in order to use the Unicode character with |\left|, |\big|, |\bigl| etc.
	(because of the internal detail being that in Xe\LaTeX\ and Lua\LaTeX,
	as this package does not change the character catcode to be active,
	it's necessary to set the |delcode| as mentioned before)

	In that case the second argument must be a single token.

	Unfortunately, the command does not always work.
\end{function}

\begin{function}{\umiRefreshDelimiterList}
	\begin{syntax}
	    |\umiRefreshDelimiterList|
	\end{syntax}
	You should normally not need this command.

	As mentioned before, in Lua\LaTeX\ once a command is redefined, the Unicode character does not automatically update.

	This command will check all the normal delimiter Unicode characters. In PDF\LaTeX\ this command does nothing.
\end{function}

\begin{variable}{ignore-refresh-delimiter-list}
	Package option.

	The command above will be run |\AtBeginDocument|. Pass this to disable it running.
	Only needed if there's some package clash.
\end{variable}

\begin{function}{\umiPatchCmdUnicodeArg,\umiUnpatchCmdUnicodeArg}
	\begin{syntax}
		|\umiPatchCmdUnicodeArg \sqrt|
		|\umiUnpatchCmdUnicodeArg \sqrt|
	\end{syntax}
	After executing this command, the command specified in the argument (|\sqrt| in this example)
	can be called with one argument being an Unicode character without needing a brace.

	(i.e. you can write |\sqrt α| instead of |\sqrt{α}|.)

	Because of implementation detail,
	\begin{itemize}
		\item |\sqrtα| (without the space between |\sqrt| and |α|) works in PDF\LaTeX\ but not Lua\LaTeX. (so this form is not recommended.)
		\item |\sqrt α| works in Lua\LaTeX\ without needing the patch. In other words, the patch does nothing in Unicode engines.
	\end{itemize}

	The command being patched must take at least one mandatory argument as the first argument, and it only affect that first argument.
	In other words, |\sqrt[3]α| cannot be patched this way unless you do e.g. |\newcommand\cbrt[1]{\sqrt[3]{#1}}| then |\umiPatchCmdUnicodeArg\cbrt|,
	then |\cbrt α| works (but |\sqrt[3]α| still doesn't).
\end{function}

\begin{function}{\umiPatchCmdUnicodeTwoArgs}
	\begin{syntax}
		|\umiPatchCmdUnicodeTwoArgs \frac|
		|\umiUnpatchCmdUnicodeArg \frac|
	\end{syntax}
	Similar to above, but for commands with (at least) two mandatory arguments.

	Only affects these 2 mandatory arguments.
\end{function}

\begin{function}{\umiPatchCmdUnicodeArgExtraGroup}
	\begin{syntax}
		|\umiPatchCmdUnicodeArgExtraGroup \Big|
	\end{syntax}
	Don't use this command unless you know exactly what you're doing.

	Similar to |\umiPatchCmdUnicodeArg|, but open an implicit group before executing anything and close the group after.

	The command being patched must take exactly one argument.

	This is useful because some \TeX\ primitives such as |^| or |\mathopen|
	requires either a single "character" or a group braced with |{...}| / |\bgroup...\egroup|.
\end{function}

\begin{variable}{ignore-patch-delimiter-commands}
	Package option. Pass this to avoid patching |\Big| etc. with the command above (only needed if there's some package clash).
\end{variable}

\begin{function}{\umiBraceNext}
	\begin{syntax}
		|\umiBraceNext {abc...} αxyz...|
	\end{syntax}
	In the example above, after some steps of execution of \TeX, the state will be
	|abc... {α}xyz...|.

	Formally: if the character following the first argument to |\umiBraceNext| is not representable in a single byte
	and the engine is not Unicode, the character will be braced, otherwise nothing happens. Then the argument
	is put back in the input stream.

	This is an internal command mainly useful for defining the command above, for example after
\begin{verbatim}
\let\oldbig\big
\def\big{\umiBraceNext{\oldbig}}
\end{verbatim}
then |\big⟨| will eventually execute |\oldbig{⟨}| which is the desired behavior (that |\oldbig| expects one braced argument).
\end{function}

\begin{variable}{ignore-patch-prime}
	Do not patch the default definition of |'| in math mode.

	By default it's patched to allow |G'²| and |G²'| to work. Only use this when there's some package clash.
\end{variable}

\begin{function}{\umiPatchPrime,\umiUnpatchPrime}
	\begin{syntax}
		|\umiPatchPrime|
		|\umiUnpatchPrime|
	\end{syntax}
	As mentioned above, by default |\umiPatchPrime| is run |\AtBeginDocument|. But it can be patched
	and unpatched manually.

	Note that it's undefined behavior if some package modifies the definition of active |'|
	while it's patched. To resolve conflict, unpatch |'|, load the package, then patch again.
\end{function}

\section{Compatibility}

This package should have tested with various \TeX\ distribution versions on Overleaf.

\PrintChanges
\PrintIndex
\Finale

\end{document}
