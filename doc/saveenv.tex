%%! TEX program = lualatex
\ProvidesFile{saveenv.tex} [2022/07/06 v0.0.0 Save environment content verbatim]
\documentclass{l3doc}
\EnableCrossrefs
\CodelineIndex
\RecordChanges
\usepackage{hyperref}
\usepackage{csquotes}
%\MakeOuterQuote{"}
\begin{document}
\GetFileInfo{\jobname.tex}

\title{\textsf{saveenv} --- Save environment content verbatim
\thanks{This file describes version \fileversion, last revised \filedate.}
}
\author{user202729%
%\thanks{E-mail: (not set)}
}
\date{Released \filedate}

\maketitle


\let\backtick=`
\MakeShortVerb{`}

\begin{abstract}
Allow users to grab environment content verbatim.
\end{abstract}

\section{Usage}

\subsection{Comparison to existing similar packages}

\begin{itemize}
	\item \pkg{scontents} package is similar; however it does not allow programmer to programmatically retrieve the macro content.
	\item \pkg{fancyvrb} and \pkg{verbatimcopy} provides some internal macro for defining similar environments, however they're internal.
	\item \pkg{filecontentsdef} provides an similar environment; however it requires \verb+\endlinechar=13+ and does not support \cs{ExplSyntaxOn} environment.
\end{itemize}

This package provides tools to create your own verbatim environments, and works in all values of \cs{endlinechar}.

Limitations: This package does not process the content line-by-line. Therefore

\begin{itemize}
	\item There might be larger memory consumption.
	\item If it's desirable to typeset the content, it's not straightforward to preserve the Sync\TeX\ information.
\end{itemize}

\subsection{Main environment}
\DescribeEnv{saveenv}
  Environment that saves its body.
  
  For example, the following code

		\begin{verbatim}
		\begin{saveenv}{\data}
		123
		456
		\end{saveenv}
		\end{verbatim}

will save the string \texttt{123\meta{newline}456} globally into \cs{data}.

The braces around \verb+{\data}+ is optional; however, in the unlikely case if \cs{endlinechar} has the \enquote{letter} catcode, it might be absorbed.

A newline is represented as a token with charcode 10 (\verb+^^J+) and catcode other.

The assignment is global, and done before the macro \cs{endsaveenv} is executed.

The data is represented as an \pkg{expl3}-string, that is, a sequence of tokens with catcode 12 (for non-space characters) or 10 (for space character).

In other words, the token list is equal to its own \cs{detokenize}.

\DescribeEnv{saveenvghost}
Similar to above; however the content inside is still typesetted/executed, and the Sync\TeX\ information is preserved.

\DescribeEnv{saveenvkeeplast}
Similar to above; however the final newline and the space characters before \texttt{\cs{end}\{\meta{environment name}\}} are preserved.

For example, the example above with \texttt{saveenv} replaced with \texttt{saveenvkeeplast}
will save the string \texttt{123\meta{newline}456\meta{newline}} into \cs{data} instead.

\subsection{Extending the environments}

All of the environments are extensible.

Follow the instruction in \url{https://tex.stackexchange.com/q/14683/250119} to define an environment based on an existing one.

For example, the following definition

		\begin{verbatim}
		\ExplSyntaxOn
		\NewDocumentEnvironment{custom}{}{
			\saveenv \data
		}{
			\endsaveenv
			% the user may want to modify \data here
		}
		\end{verbatim}
will define an environment \env{custom} that wraps \env{saveenv}.

\subsection{Common issues}

\begin{itemize}
	\item The \cs{endlinechar} must not be tokenized in advance. This might happen when for example your environment looks ahead for optional argument.
		See \url{https://tex.stackexchange.com/q/649331/250119} for an example.

		Alternatively, you can 
		\begin{itemize}
			\item manually remove the end line token,
			\item set \verb+\endlinechar=-1\relax+ before calling \cs{saveenv}
			\item (optional) reset the value of \cs{endlinechar} afterwards.
				
		\end{itemize}

	\item Note that there must be nothing after the \texttt{\cs{begin}\{\meta{environment name}\}} or the \texttt{\cs{end}\{\meta{environment name}\}} line.
\end{itemize}



\StopEventually{%
  \PrintChanges
  \PrintIndex
}
\Finale
\end{document}

