%! TEX program = lualatex
\ProvidesFile{typstmathinput.tex}[2023/03/28 0.0.0 Typst-style math input format]
\RequirePackage{fvextra}
\documentclass{l3doc}
\EnableCrossrefs
\CodelineIndex
\fvset{breaklines=true,gobble=0,tabsize=4,frame=single,numbers=left,numbersep=3pt}
\AtBeginDocument{\DeleteShortVerb\"}  % https://tex.stackexchange.com/a/650966/250119
\MakeOuterQuote{"}
\usepackage{hyperref}
%\usepackage{typstmathinput}
\usepackage{saveenv}
\usepackage{silence}
\usepackage{typstmathinput}
\usepackage{unicode-math}
%\WarningFilter{latex}{Marginpar on page}
\begin{document}
\typstmathinputenable{$}
\GetFileInfo{\jobname.tex}

\title{\pkg{\jobname} --- \fileinfo
\thanks{This file describes version \fileversion, last revised \filedate.}
}
\author{user202729}
\date{Released \filedate}

\maketitle

\section{Introduction}

Typst is a new typesetting system that aims to be as powerful as \LaTeX\ while being much easier to learn.

Unfortunately, at the time of writing Typst has several issues
(refer to \url{https://github.com/typst/typst/issues})
that may make it unusable for some purposes.

As such, this package allows the user to enter \emph{math formulas} in Typst input format
while using \LaTeX.

This package requires the engine Lua\LaTeX.

\section{Usage}

First include the package:
\begin{verbatim}
\usepackage{typstmathinput}
\end{verbatim}

Then you can type math input similar to Typst: the following code, taken from \href{https://typst.app/docs/reference/math/}{Typst tutorial} on math mode

\newenvironment{demo}{\saveenv\tmp}{\endsaveenv
\scantokens\expandafter{\expanded{\string\begin{verbatim}^^J\tmp\string\end{verbatim}}}
gives the result:

\begingroup
\scantokens\expandafter{\tmp}
\endgroup
}

\begin{demo}
$ A = pi r^2 $
$ "area" = pi dot.op "radius"^2 $
$ cal(A) :=
    { x in RR | x "is natural" } $
$ x < y => x gt.eq.not y $
$ sum_(k=0)^n k
    &= 1 + ... + n \
    &= (n(n+1)) / 2 $
$ frac(a^2, 2) $
$ vec(1, 2, delim: "[") $
$ mat(1, 2; 3, 4) $
$ lim_x =
    op("lim", limits: #true)_x $
\end{demo}

Currently, some preprocessing is done before passing the data to Typst, in order to handling
superscript/subscript characters and square root Unicode sign correctly,
refer to Typst feature requests \href{https://github.com/typst/typst/issues/592}{\#592}
and \href{https://github.com/typst/typst/issues/516}{\#516} for details.

When these feature requests are implemented, this package should be amended to remove the patches.

Some options are possible:

\begin{variable}{cache-location=}
    Specify the location of the cache.
    If not passed or empty, defaults to some location in the temporary directory.
\end{variable}

\begin{variable}{cache-format=}
    Specify the format of the cache. Possible formats are |json| and |shelve|.
    
    The default is |shelve|, but the problem is that if multiple concurrent runs are ongoing,
    the error message "Resources temporarily available" may be raised.

    There doesn't appear to be a good solution against it.
\end{variable}

\begin{variable}{watch-template-change=}
    If |true|, compare if the template has changed since the last run.

    This option is for the package developer only.
    Here the template refers to the \file{typstmathinput-template.typ} file
    that should be included in the package, which the user should not modify.
\end{variable}



\section{Limitations}

Note that this only handle math mode, that is, you cannot use |*text*| for bold, etc.
Package \pkg{markdown} might be of interest.

It does not work inside environments such as |\begin{align*}...\end{align*}|,
or |\[...\]|,
however you can manually trigger the parsing, and you can use Typst's dedicated syntax for math alignment.

Note that if you use packages such as |fancyvrb|
or |csquotes| and make \verb+|+ or \texttt{\string"} have special meaning
(verbatim or smart quote respectively),
they might not work in math environments.

|$$...$$| is not supported (although this is not used in Typst anyway, put a space around the dollar to
use display mode)

This calls Typst executable plus |pdftotext| for each formula being defined, which makes it extremely slow --
although some caching is in effect, so it will only be slow the first time.

%Only a small number of symbols are defined, although this is customizable, refer to the section below.

%\LaTeX\ names of symbols are still usable, for example |$hbar$|.

%It's possible to use unbalanced braces inside math

\section{Advanced usage}

You can optionally enable or disable the special handling:

\DescribeMacro{\typstmathinputenable}
\DescribeMacro{\typstmathinputdisable}
These macros can be used to enable or disable the special handling of the |$...$|.

\begin{demo}
With Typst-input-math:
$P => sin theta + cos phi = tan tau$
\typstmathinputdisable{$}

Without Typst-input-math:
$P \implies \sin \theta + \cos \phi = \tan \tau$
\end{demo}

\DescribeMacro{\typstmathinputtext}
By default, it's an alias for \pkg{amsmath}'s |\text| command. You can redefine it
to customize the behavior.

\begin{demo}
Normal behavior: $"hello world"$

\renewcommand\typstmathinputtext[1]{\textcolor{red}{\text{#1}}}
Redefined: $"hello world"$
\end{demo}

\DescribeMacro{\typstmathinputextrapreamble}
A string consist of Typst code if the user want to define custom things.

The \pkg{saveenv} package may be useful for this. Example usage:
\begin{verbatim}
\begin{saveenv}{\typstmathinputextrapreamble}
#let abc = 1
\end{saveenv}
\end{verbatim}

Note that explicit double-hash here may be converted to single-hash, which can be a bit confusing.


%\DescribeMacro{typstmathinput.parse()}
%This Lua function can be used to parse a Typst-style input to give TeX-style input.
%For example the Lua code
%\begin{verbatim}
%print(typstmathinput.parse("sin theta", false))
%\end{verbatim}
%will print |\sin \theta |.
%
%The second argument is a boolean that indicates whether the input is in display mode.
%
%\DescribeMacro{\typstmathinputdefine}
%Define new symbols in Typst. Use as follows:
%\begin{verbatim}
%\typstmathinputdefine{planck.reduce}{\hbar}
%\end{verbatim}
%The first argument is the symbol name in Typst environment, the second argument is some \LaTeX\ code.
%
%You can also use an Unicode character as the first argument.


\PrintChanges
\PrintIndex
\Finale
\end{document}
