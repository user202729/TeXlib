\ProvidesFile{utf8y.tex}[2022/09/17 0.0.0 Unicode characters defined in ucs/utf8x]
\documentclass{l3doca}
\begin{document}
\GetFileInfo{\jobname.tex}
\title{\pkg{\jobname} --- \fileinfo
\thanks{This file describes version \fileversion, last revised \filedate.}
}
\author{user202729}
\date{Released \filedate}
\maketitle

\begin{abstract}
	Unicode characters defined in ucs/utf8x.
\end{abstract}

\section{Motivation}

\pkg{utf8x} is mostly the only choice for Unicode-input for math formula nowadays, but it's
rather annoying that it is
\begin{itemize}
	\item incompatible with many, many packages (\url{https://tex.stackexchange.com/q/13067/250119}), and
	\item is only usable in PDFLaTeX.
\end{itemize}

If you do low-level TeX programming, you may also notice that
\pkg{ucs}-declared macros does not work very well peeking forward
(for example, in the following code
\begin{verbatim}
\def\mymacro#1{...}
\DeclareUnicodeCharacter{8594}{\mymacro}

→{123}
\end{verbatim}
where |8594| is the Unicode value of the character |→| in decimal,
then the argument will not be equal to |123| when \pkg{ucs} is used)
which is rather limiting.

As such, this package allows entering Unicode character in the input
without using \pkg{unicode-math}.

For Unicode engines, only math-mode characters are defined.

\section{Features}

Mostly, all the characters that \pkg{ucs} supports are supported.

Just like with the \pkg{ucs} package, sometimes you may get spurious "undefined control sequence" error, which happens
because you may need to load some additional packages that defines particular commands to use the character (for example load \pkg{amssymb}
so that |\emptyset| is defined to use $\emptyset$)

The following features are not supported:

\begin{itemize}
	\item Combining characters: while it can be implemented, it requires either 
		\begin{itemize}
			\item scanning through the whole file in advance (thus needs something like \pkg{rescansync}
				to avoid losing Sync\TeX\ data, but then the current approach only works in \LuaLaTeX\ so it's useless in this case), or
			\item wrapping all the text that possibly use combining character within some command (and lose Sync\TeX\ data in that block)
		\end{itemize}
		In either case, it's not a very nice solution.

	\item Most of the time |\ensuremath| around some characters are removed, as such they're only usable in math mode.
\end{itemize}

\section{Additional feature}

In addition, if you type consecutive superscript/subscript characters in the source code, it will be converted to the correct sequence.

\section{Implementation remark}

Although all characters are ported from \pkg{ucs} package, only the characters that the package author frequently use is actually tested.

First the bundle is downloaded from 
\url{https://ctan.org/pkg/ucs}, then use the \TeX\ file to generate the output.

For characters where the standard \pkg{inputenc} already defines a text-mode meaning, the new definition is defined to use |\ifmmode|.
(thus, for example $\times$ is defined to be |\texttimes| in text mode, and |\times| in math mode)

|\relax| is inserted before |\ifmmode| in such cases, to avoid constructs based on |\halign| expanding the head of a table cell.
Refer to  the second double-bend paragraph in page 240 of the \TeX book if you don't understand what this sentence is about.

\PrintIndex
\end{document}
