%%! TEX program = lualatex
\ProvidesFile{precattl.tex} [2022/07/07 v0.0.0 ]
\documentclass{l3doc}
\EnableCrossrefs
\CodelineIndex
\RecordChanges
\fvset{gobble=0,tabsize=4,frame=single,numbers=left,numbersep=3pt}
\usepackage{hyperref}
\usepackage{csquotes}

%\MakeOuterQuote{"}
\begin{document}
\GetFileInfo{\jobname.tex}

\title{\textsf{precattl} --- Prepare special catcodes from token list
\thanks{This file describes version \fileversion, last revised \filedate.}
}
\author{user202729%
%\thanks{E-mail: (not set)}
}
\date{Released \filedate}

\maketitle

\begin{abstract}
Allow users to write code that contains tokens with unusual catcodes.
\end{abstract}

\section{Motivation}

This package allows developers to quickly prototype writing code by specifying the exact token list to be executed.

For example, classically if you want to define a space gobbler you would do this
\begin{verbatim}
\lowercase{\def\spacegobbler} {}
\end{verbatim}
which is equivalent to the token list |\def\spacegobbler|\meta{explicit space token}\verb+{}+.

In such simple cases there's negligible benefit in using this package; nevertheless in some more complex cases such as the following...
\begin{verbatim}
\begingroup
\lccode `?=`_
\lowercase{
	\endgroup
	\peek_analysis_map_inline:n {
		\expandafter \string #1
		\peek_analysis_map_break:
	}
	\c_begin_group_token ?
}

\lowercase{\lowercase{\def\twospacegobbler} } {}

\catcode`\^^M\active
\edef^^M{\string^^M}
\end{verbatim}
it would be beneficial to have simpler construct to express the code.

A side benefit is that there's no need to do explicit |\begingroup| ... |\endgroup| (and keep track of when |\endgroup| should be executed),
and no risk of accidentally pretokenizing the token list/it can be used inside argument of anything.

\section{Limitation}

\begin{itemize}
	\item There's some performance hit. I think it's about 5 times slower than |\tl_analysis_map_inline:nn| on the same token list.

		Nevertheless this is not a big problem, as it's possible to precompile the token list and include it in the generated \texttt{.sty} file.
	\item It's not allowed to pass |\outer| tokens into the macro directly; however, generating |\outer| tokens with |\cC| and |\cA| is supported.
\end{itemize}


\section{Syntax}


\begin{function}{\precattl_exec:n}
  \begin{syntax}
	  \cs{precattl_exec:n} \Arg{token list}
  \end{syntax}
  Executes the \Arg{token list} after preprocessing it in the following manner:

  \begin{itemize}
	  \item |\cC|\Arg{control sequence name} is replaced by that control sequence.
	  \item |\cFrozenRelax| is replaced with the frozen relax control sequence.
	  \item |\cA|\Arg{tokens}, |\cO|\Arg{tokens}, etc. are replaced with sequence of tokens with catcode active/other respectively.
  \end{itemize}
\end{function}
  In particular, \Arg{tokens} or \Arg{control sequence name} might consist of normal \enquote{characters} or control sequences.
  Control sequences will have its csname appended. (nevertheless don't use the null control sequence here)

  The full list of supported catcodes is the same as \pkg{l3regex} package: 
  |\cB|, |\cE|, |\cM|, |\cT|, |\cP|, |\cU|, |\cD|, |\cS|, |\cL|, |\cO|, |\cA|.

  As an example, the examples above can be executed as following:
\begin{verbatim}
\precattl_exec:n {
	\def\spacegobbler\cS\ {}

	\peek_analysis_map_inline:n {
		\expandafter \string #1
		\peek_analysis_map_break:
	}
	\c_begin_group_token \cO\_

	\def\twospacegobbler\cS\ \cS\ {}

	\def \cA\^^M {\cO\^^M}
}
\end{verbatim}
More details: the token list following one of the |\c|\meta{character} token might either be a single token, or a braced group. In the latter case their string representation without any |\escapechar| will be concatenated together.

For example:

\begin{itemize}
	\item |\cL{12~34}| in \verb+\ExplSyntaxOn+ mode results in the 5 tokens with catcode letter and char code 1, 2, \meta{space}, 3, 4 respectively.
	\item |\cO\abc| results in 3 tokens |abc| with catcode other.
	\item |\cO{\ab\cd\ef}| results in 6 tokens |abcdef|. Note that this is different from the |\detokenized| value (that is, the spaces are removed) even when |\escapechar=-1|.
	\item |\cC{\ab\cd\ef}| results in the control sequence |\abcdef|.
	\item |\cO\\| results in a single token with char code |\| and catcode other.
	\item |\cA\^^M| results in a single token with catcode active and char code 13. (because normal \TeX\ processing rules transform the token |\^^M| to a control sequence with name length = 1 before passing it to the function \cs{precattl_exec:n})
\end{itemize}

Remark: don't put explicit space token right after a |\cO| or similar. (nevertheless this is nontrivial to trigger because of how \TeX\ works normally)

Warning: a  \verb*|\ | at the end of a line will be interpreted as an escape sequence containing the |\endlinechar|.

\DescribeEnv{precattlenv}
Same as above, but the content is put inside an environment. Use it like this.
\begin{verbatim}
\begin{precattlenv}
\def\spacegobbler\cS\ {}
\end{precattlenv}
\end{verbatim}
It's guaranteed to execute the content outside a group; that is, any local assignment made inside the environment persist outside it as well.

\begin{function}{\precattlExec}
\LaTeX2-style synonym for the function above.
\end{function}

\begin{function}{\precattl_set:Nn}
  \begin{syntax}
	  \cs{precattl_set:Nn} \meta{tl var} \Arg{token list}
  \end{syntax}
  Same as above, but instead of executing the processed \Arg{token list}, the result is stored into \meta{tl var}.

  Mostly equivalent to |\precattl_exec:n {\tl_set:Nn| \meta{tl var} \Arg{token list}|}|.
\end{function}

\PrintChanges
\PrintIndex
\Finale
\end{document}

