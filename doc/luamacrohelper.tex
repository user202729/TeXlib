\ProvidesFile{luamacrohelper.tex}[2022/07/16 0.0.0 Tools to help with writing TeX macros in Lua]
\RequirePackage{fvextra}
\documentclass{l3doc}
\EnableCrossrefs
\CodelineIndex
\fvset{breaklines=true,gobble=0,tabsize=4,frame=single,numbers=left,numbersep=3pt}

\AtBeginDocument{\DeleteShortVerb\"}  % https://tex.stackexchange.com/a/650966/250119
\MakeOuterQuote{"}

\NewDocumentEnvironment{luafn}{}{\function}{\endfunction}  % for semantic difference only, currently there's no functional difference

\newcommand\reg{régime}

\begin{document}
\GetFileInfo{\jobname.tex}

\title{\pkg{\jobname} --- Tools to help with writing \TeX\ macros in Lua
\thanks{This file describes version \fileversion, last revised \filedate.}
}
\author{user202729}
\date{Released \filedate}

\maketitle

\begin{abstract}
Tools to help with writing \TeX\ macros in Lua.
\end{abstract}

\section{Motivation}
Currently, \LuaTeX\ embeds a Lua interpreter in \TeX\ which allows
efficient interleaving of Lua code and \TeX\ code;
however if you want to write \TeX\ macros from Lua,
the code is not very pretty because the primitives \LuaTeX\ provides
has some quirks that need workarounds in common cases.

This package provides
\begin{itemize}
	\item common library functions for creating \TeX\ macros in Lua
	\item convenient syntactic sugar
	\item some additional debug traceback capacities.
\end{itemize}

Unless specifically noted, any case where Lua traceback is not printed if the error happens while some Lua code is running is considered a bug in the package.

\section{Loading the package}

Execute |\usepackage{luamacrohelper}| as usual.

Then the API is provided by, executing from Lua, |local L = require "luamacrohelper"|.
In the following documentation |L| will be used to denote the library table.

Strictly speaking, the |\usepackage| line is not necessary;
however the package does some initialization on first load.

\section{Convention}

Throughout this document:
\begin{itemize}
	\item \meta{token object} is used to represent a Lua built-in token object.
	\item \meta{token list} is used to represent a Lua integer-indexed table of \meta{token object}s.
	\item \meta{function} is used to represent a Lua function object.
\end{itemize}

\section{\label{sec:possibleissues} Possible issues}

If you encounter the messages \\
|TeX capacity exceeded, sorry [text input levels=15].| \\
or \\
|TeX capacity exceeded, sorry [input stack size=5000].| \\
you may consider the following:
\begin{itemize}
	\item Use the wrapper of this package (|L.runlocal|, |L.runpeek|, |L.scan_toks|, |L.get_argument| etc.)
		instead of the \LuaTeX\ primitives (|token.runtoks|, |token.scan_toks| etc.)
	\item See also the documentation of |L.unwind_input_stack|.
\end{itemize}


Refer to \url{https://tex.stackexchange.com/q/640922/250119} for more details.

\section{Performance note}

Because of the additional error-check requirement, the package is unfortunately not very fast.

Suggestions for improving the performance while not sacrificing error-checking capabilities are welcome.

\section{Documentation note}

The index doesn't really work as expected, as this documentation was written with \pkg{l3doc} which is intended
to document \pkg{expl3}-style macros...

\section{API documentation}

\begin{luafn}{L.luadef}
	\begin{syntax}
		L.luadef(\meta{function name}, \meta{function}, \meta{optional prefix})
	\end{syntax}
	Define a \TeX\ macro similar to |\def|. In \LuaTeX, this will expand in one step to its result.

	Explicitly providing argument is not supported; instead, get the arguments inside the function.

	The third argument is optional, it's a table such as |{protected=1, long=1}| -- the table keys are used as prefixes for the new control sequence.
\end{luafn}

\begin{luafn}{L.protected_luadef,L.protected_long_luadef}
	\begin{syntax}
		L.protected_luadef(\meta{function name}, \meta{function})
		L.protected_long_luadef(\meta{function name}, \meta{function})
	\end{syntax}
	Simple wrapper for the above.
\end{luafn}


\begin{luafn}{L.get_argument}
	\begin{syntax}
		L.get_argument() -> \meta{token list}
	\end{syntax}
	Get an argument in the input stream.
\end{luafn}

\begin{luafn}{L.get_argumente}
	\begin{syntax}
		L.get_argumente() -> \meta{token list}, bool
	\end{syntax}
	Same as above. The second argument determines whether the argument is braced.
\end{luafn}

\begin{luafn}{L.get_argument_braced,L.get_argument_unbraced}
	\begin{syntax}
		L.get_argument_braced() -> \meta{token list}
		L.get_argument_unbraced() -> \meta{token list}
	\end{syntax}
	Same as above, enforce the argument to be braced/unbraced.
\end{luafn}


\begin{luafn}{tex.sprint,tex.print}
	\begin{syntax}
		tex.sprint(\meta{arguments})
		tex.print(\meta{arguments})
	\end{syntax}
	This \LuaTeX\ built-in can be used to print \TeX\ code. Refer to \LuaTeX\ documentation for details which arguments are supported.

	Note that this is not fully supported and may cause some issues! Use with care.

	Currently only |L.runlocal| fully support this.
\end{luafn}

These functions can be used to replace \TeX\ primitive |\def| statements as following:

\begin{verbatim}
L.def("mymacro", function()
	local a=L.get_argument()
	local b=L.get_argument()
	tex.sprint("(")
	tex.sprint(a)
	tex.sprint(",")
	tex.sprint(b)
	tex.sprint(")")
end)
% \def\mymacro#1#2{(#1,#2)}
\end{verbatim}

\begin{luafn}{L.tl_equal}
	\begin{syntax}
		L.tl_equal(\meta{token list}, \meta{token list}) -> bool
	\end{syntax}
	Check if two token lists are equal.
\end{luafn}

\begin{luafn}{L.token_equal}
	\begin{syntax}
		L.token_equal(\meta{token object}, \meta{token object}) -> bool
	\end{syntax}
	Check if two token objects are equal. (internally this function compares their |.tok| value)
\end{luafn}


\begin{luafn}{L.T}
Create a token. See example for details how to use it. (remember that |T "{"| is Lua shorthand for function call |T("{")|)
\begin{verbatim}
local T=L.T
T "{"      -- create the begin-group character token  (using current catcode of {)
T("{", 1)  -- same as above, explicitly provide the catcode
T(string.byte "{", 1)  -- same as above, provide the character code as an integer. Note that string.byte is not UTF8-aware
T.expandafter     -- create the control sequence token \expandafter
T["expandafter"]  -- same as above
\end{verbatim}
Unlike |token.create()|, this supports creating the null control sequence |T[""]|,
as well as control sequences that is not in the hash table (in which case they will be added to the hash table).
\end{luafn}

\begin{luafn}{L.bgroupT, L.egroupT, L.mathT, L.alignmentT, L.paramT, L.superscriptT, L.subscriptT, L.spaceT, L.letterT, L.otherT, L.activeT}
	\begin{syntax}
L.bgroupT(...)
L.egroupT(...)
L.mathT(...)
L.alignmentT(...)
L.paramT(...)
L.superscriptT(...)
L.subscriptT(...)
L.spaceT(...)
L.letterT(...)
L.otherT(...)
L.activeT(...)
\end{syntax}
For convenience, those functions are also provided for creating character tokens with given catcode.
The argument is either a string or a character code, similar to first argument of |T|.
\end{luafn}

\begin{luafn}{L.begin_groupT, L.end_groupT, L.math_toggleT, L.parameterT, L.math_superscriptT, L.math_subscriptT}
	\begin{syntax}
L.begin_groupT(...)
L.end_groupT(...)
L.math_toggleT(...)
L.parameterT(...)
L.math_superscriptT(...)
L.math_subscriptT(...)
\end{syntax}
Verbose (\pkg{expl3}-naming style) aliases for the functions above.
\end{luafn}

\begin{luafn}{L.is_explicit_character}
	\begin{syntax}
L.is_explicit_character(\meta{token object}) -> bool
\end{syntax}
Check if a token object is an explicit character. Might fail for some special internal tokens such as the internal |\endlocalcontrol|.
\end{luafn}

\begin{luafn}{L.get_catcode}
	\begin{syntax}
L.get_catcode(\meta{token object}) -> int
\end{syntax}
Get the catcode. \meta{token object} must be an explicit character token.
\end{luafn}

\begin{luafn}{L.get_charcode}
	\begin{syntax}
L.get_charcode(\meta{token object}) -> int
\end{syntax}
Get the charcode. \meta{token object} must be an explicit character token.
\end{luafn}


\begin{luafn}{L.is_bgroup, L.is_egroup, L.is_math, L.is_alignment, L.is_param, L.is_superscript, L.is_subscript, L.is_space, L.is_letter, L.is_other, L.is_active}
	\begin{syntax}
L.is_bgroup(\meta{token object})       -> bool
L.is_egroup(\meta{token object})       -> bool
L.is_math(\meta{token object})         -> bool
L.is_alignment(\meta{token object})    -> bool
L.is_param(\meta{token object})        -> bool
L.is_superscript(\meta{token object})  -> bool
L.is_subscript(\meta{token object})    -> bool
L.is_space(\meta{token object})        -> bool
L.is_letter(\meta{token object})       -> bool
L.is_other(\meta{token object})        -> bool
L.is_active(\meta{token object})       -> bool
\end{syntax}
Check if a token object is an explicit character of specified catcode.
\end{luafn}


\begin{luafn}{L.E3}
	\begin{syntax}
L.E3.\meta{name}.\meta{argspec}
L.E3.\meta{name}()
L.E3.\meta{name}(\meta{argspec})
L.E3(\meta{name}).\meta{argspec}
L.E3(\meta{name})(\meta{argspec})
L.E3(\meta{name}, \meta{argspec})
\end{syntax}
Convenience function to create an \pkg{expl3}-style control sequence.
For example
\begin{itemize}
	\item |E3.use.n| creates the control sequence |\use:n|,
	\item |E3.tl_map_break()| creates the control sequence |\tl_map_break:|.
\end{itemize}
The syntax is flexible, either indexing or calling is supported (although if you pass a string in, it might be more convenient to simply evaluate |T["use:n"]|)
\end{luafn}

\begin{luafn}{L.runlocal}
	\begin{syntax}
		L.runlocal(\meta{token list})
		L.runlocal(\meta{function})
	\end{syntax}
	Execute a token list, it should not look forward in the input stream, and should not leave any tokens in the input stream.

	Wrapper over |tex.runtoks()|.

	Alternatively a function that prints out a token list can be passed in. Any return value of the function will be returned by |runlocal|.

	For example
\begin{verbatim}
L.runlocal {T.def, T.mymacro, bgroup, egroup}
\end{verbatim}

	This function will print a Lua traceback if the \TeX\ code inside gives an error while executing, as well as diagnosing some simple cases of the executing token list is nonlocal
	(although the diagnosis might not always be correct).
\end{luafn}

\begin{luafn}{L.runpeek}
	\begin{syntax}
		L.runpeek(\meta{token list})
		L.runpeek(\meta{function})
	\end{syntax}
	Execute a token list. It's allowed to look forward in the input stream, or leave tokens in the input stream.

	After done executing, a token |\endlocalcontrol| must be executed.

	For example, after the following is executed
\begin{verbatim}
L.runpeek {T["exp_args:Nx"], T.endlocalcontrol, bgroup, T.sometoken, egroup}
\end{verbatim}
the tokens |{|\meta{\texttt{x}-expansion of \texttt{\textbackslash sometoken}}|}| is left ahead in the input stream.

	Note that the \meta{function} should preferably not use |tex.sprint|, as it might cause some issue.
\end{luafn}

\begin{luafn}{L.token_of}
	\begin{syntax}
		L.token_of(\meta{function}, \meta{optional prefix})
	\end{syntax}
	Return a token has is |luadef|-ed to the provided function.
\end{luafn}

\begin{luafn}{L.get_next}
	\begin{syntax}
		L.get_next() -> \meta{token object}
	\end{syntax}
	Wrapper for token.get_next() but handles never seen control sequence correctly (i.e. return the control sequence, add to hash table)
\end{luafn}

\begin{luafn}{L.get_nexte}
	\begin{syntax}
		L.get_nexte() -> \meta{token object}, bool
	\end{syntax}
	Same as above.

	There's a second result value, whether the obtained token is a notexpanded token. Does not always detect successfully.
\end{luafn}

\begin{luafn}{L.futurelet}
	\begin{syntax}
		L.futurelet(\meta{token object})
	\end{syntax}
	Assign \meta{token object} to the meaning of the next token in the input stream, like the primitive |\futurelet|.

	Note that this function as well as the function below might remove the notexpanded status of the next token.
\end{luafn}

\begin{luafn}{L.futureletafter}
	\begin{syntax}
		L.futureletafter(\meta{token object})
	\end{syntax}
	Assign \meta{token object} to the meaning of the next-next token in the input stream. Uses the "|\afterassignment|+|\futurelet|" trick, details in \url{https://tex.stackexchange.com/a/19769/250119}.
\end{luafn}

\begin{luafn}{L.expandonce}
	\begin{syntax}
		L.expandonce()
	\end{syntax}
	Expand the following tokens in the input stream once.
\end{luafn}

\begin{luafn}{L.exp_o,L.exp_oo}
	\begin{syntax}
		L.exp_o(\meta{token list}) -> \meta{token list}
		L.exp_oo(\meta{token list}) -> \meta{token list}
	\end{syntax}
	Return the |o|-expansion of \meta{token list} and the twice-expansion, respectively.

	The provided token list does not need to be balanced, only the result need to.

	Note that in certain cases (such as with notexpanded token) the |exp_oo| result might not be equal to
	|exp_o| applied twice; also, the intermediate result of |exp_oo| does not need to be balanced either.
\end{luafn}

\begin{luafn}{L.exp_x}
	\begin{syntax}
		L.exp_x(\meta{token list}) -> \meta{token list}
	\end{syntax}
	Return the |x|-expansion of \meta{token list}.
\end{luafn}

\begin{luafn}{L.exp_e}
	\begin{syntax}
		L.exp_e(\meta{token list}) -> \meta{token list}
	\end{syntax}
	Return the |e|-expansion of \meta{token list}. This always use the |\expanded| primitive.
\end{luafn}


\begin{luafn}{L.tl_to_str}
	\begin{syntax}
		L.tl_to_str(\meta{token list}) -> string
	\end{syntax}
	Convert a token list consist of detokenized characters to Lua (UTF-8) string.

	If the token list is not guaranteed to contain only detokenized characters, see |L.detokenize_str| instead.
\end{luafn}

\begin{luafn}{L.str_to_tl}
	\begin{syntax}
		L.str_to_tl(string) -> \meta{token list}
	\end{syntax}
	Convert a (UTF-8) string to a token list consist of detokenized characters.

	See also: |L.tokenize|.
\end{luafn}

\begin{luafn}{L.detokenize}
	\begin{syntax}
		L.detokenize(\meta{token list}) -> \meta{token list}
	\end{syntax}
	Return the result of applying the \TeX\ primitive |\detokenize| on the token list.
\end{luafn}

\begin{luafn}{L.detokenize_str}
	\begin{syntax}
		L.detokenize_str(\meta{token list}) -> string
	\end{syntax}
	Return the result of applying the \TeX\ primitive |\detokenize| on the token list, as a Lua string.

	Identical to \texttt{L.tl_to_str(L.detokenize(\meta{token list}))}.
\end{luafn}

\begin{luafn}{L.meaning,L.meaning_str}
	\begin{syntax}
		L.meaning(\meta{token object}) -> \meta{token list}
		L.meaning_str(\meta{token object}) -> string
	\end{syntax}
	Return the result of applying the \TeX\ primitive |\meaning| on the token object.
\end{luafn}

\begin{luafn}{L.stringify,L.stringify_str}
	\begin{syntax}
		L.stringify(\meta{token object}) -> \meta{token list}
		L.stringify_str(\meta{token object}) -> string
	\end{syntax}
	Return the result of applying the \TeX\ primitive |\string| on the token object.
\end{luafn}

\begin{luafn}{L.is_token}
	\begin{syntax}
		L.is_token(\meta{object}) -> bool
	\end{syntax}
	Check if the provided object is a \meta{token object}.
\end{luafn}

\begin{luafn}{L.is_tl}
	\begin{syntax}
		L.is_tl(\meta{object}) -> bool
	\end{syntax}
	Check if the provided object is a \meta{token list}.
\end{luafn}

\begin{luafn}{L.get_macro}
	\begin{syntax}
		L.get_macro(\meta{token object}) -> \meta{token list}
	\end{syntax}
	Get the content of a \pkg{expl3} \meta{tl var}.

	For example after |\tl_set:Nn \__a {123}|, |L.get_macro(T.__a)| returns the \meta{token list} "123".

	It's not supported to pass a non-\meta{tl var} in.
\end{luafn}

\begin{luafn}{L.set_macro}
	\begin{syntax}
		L.set_macro(\meta{token object}, \meta{token list})
	\end{syntax}
	Set the content of a \pkg{expl3} \meta{tl var}.
\end{luafn}

\begin{luafn}{L.tokenize}
	\begin{syntax}
		L.tokenize(string, setup) -> \meta{token list}
	\end{syntax}
	Tokenize some \TeX\ code under the current catcode \reg. The result must be a balanced token list.

	\meta{setup} is an optional string/token list that consist of \TeX\ code to setup the catcode \reg.
\end{luafn}

\begin{luafn}{L.unwind_input_stack}
	\begin{syntax}
		L.unwind_input_stack()
	\end{syntax}
	\textbf{This is an internal function.}
	If you find yourself need to use this function and does not use \LuaTeX\ primitives such as |tex.sprint| yourself,
	report a bug to this package.

	Unwinds the input stack. Refer to Section~\ref{sec:possibleissues} for more information.

	This only works for token-list-type input stack, not buffer-type input stack such as those created with |tex.sprint|.
\end{luafn}

\PrintIndex
\end{document}
