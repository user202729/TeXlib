
\newcommand\DescribeOption{\DescribeMacro}  % hack
\newcommand\DescribePython{\DescribeMacro}  % hack



You need to install Python \pkg{daemoniker} and \pkg{psutil} package.

Just like \pkg{Perl\TeX} or \pkg{PyLua\TeX}, this only requires a single run, and variables are persistent throughout the run.

Unlike \pkg{Perl\TeX} or \pkg{PyLua\TeX}, there's no restriction on compiler or script required to run the code.

\section{Usage}

\subsection{Package options}

\DescribeOption{outputdir=}

Specify the output directory if it's not the default value.

The value will be x-expanded.

Using the \pkg{outputdir} package, it's possible to determine the output directory automatically, subject to restrictions.

\DescribeOption{mode=}

Method to communicate between \TeX\ and Python. Usually the method will be automatically detected.

\subsection{Commands and environments}

The interface mimics those in popular packages such as \pkg{Python\TeX} or \pkg{PyLua\TeX}.

\subsubsection{Print to \TeX}\label{print-to-TeX}

\DescribePython{pythonimmediate.print()}
\DescribePython{pythonimmediate.file}

These functions are used in |\pyc| command or |pycode| environment.

Using |print()| function in Python will print to the console (\TeX\ standard output).
In order to print \TeX\ code to be executed, you can do one of
\begin{verbatim}
pythonimmediate.print(...)
print(..., file=pythonimmediate.file)  # note that in quiet environments, pythonimmediate.file is None, this will print to stdout instead of suppress output
with contextlib.redirect_stdout(pythonimmediate.file):
	print(...)
\end{verbatim}

All output will be buffered until the whole Python code finishes executing.
In order to typeset the text immediately use one of the advanced commands.

\subsubsection{Inline commands}

\DescribeMacro{\py}

Evaluate some Python expression, and typesets the result as a string.

The value can only be used to typeset text, it must not be used to pass "values" to other \LaTeX\ commands.

The code will be tokenized.

Example:
\begin{verbatim}
The value of $1+1$ is $\py{1+1}$.

\setcounter{abc}{\py{1+1}}  % illegal, the result (2) can only be used to typeset text
\end{verbatim}

\DescribeMacro{\pyc}

Execute some Python code.

The macro is not expandable -- roughly speaking, you can only use this at "top level".

Any output (as described in \ref{print-to-TeX}) will be typesetted.

The code will be tokenized.

\DescribeMacro{\pycq}

Same as above, but output will not be typesetted.

\DescribeMacro{\pyfile}
\DescribeMacro{\pyfileq}

Same as above.

\DescribeMacro{\pys}

Performs "string interpolation", the same way as \pkg{Python\TeX}.

\subsubsection{Environments}

\DescribeEnv{pycode}

Special note: white spaces at the end of lines are preserved.

Any output (as described in \ref{print-to-TeX}) will be typesetted.

\DescribeEnv{pycodeq}

\DescribeEnv{pysub}


\subsubsection{Python interface}

There are a lot of Python commands provided.

\DescribePython{pythonimmediate.print()}
\DescribePython{pythonimmediate.file}
Described above.

\DescribePython{pythonimmediate.newcommand()}
\DescribePython{pythonimmediate.renewcommand()}

Same as \LaTeX's |\newcommand| and |\renewcommand|. Can be used as follows:

\begin{verbatim}
@newcommand
def function():
	...
# define |\function| in TeX

@newcommand("controlsequencename")
def function():
	...
# define |\controlsequencename| in TeX

def function():
	...
newcommand("controlsequencename", function)
\end{verbatim}

\DescribePython{pythonimmediate.get_argument_detokenized()}
\DescribePython{pythonimmediate.get_optional_argument_detokenized()}
\DescribePython{pythonimmediate.get_verbatim_argument_detokenized()}
\DescribePython{pythonimmediate.get_multiline_verbatim_argument_detokenized()}

\section{Implementation note}

Communication between \TeX\ and Python are done by opening two pseudo-files from the output of a Python process |textopy|
(similar to |\ior_shell_open:Nn|)
and to the input of another Python process |pytotex| (this would be |\iow_shell_open:Nn|, if \LaTeX3 have such a function).

There are various methods for the 2 Python child processes to communicate with each other.
After some initial bootstrapping to setup the communication, we can consider only the |textopy| script, the other
merely serves as the bridge to send input to \TeX.

The communication protocol is a little complicated, since it must support nesting bidirectional execution of \TeX\ and Python.

Besides, I believe it's not possible to make a "background listener" on the \TeX\ side, so it must keep track of whether a command should be read from Python and executed.

We have the following 3 internal macros/environments to execute Python code at the lowest level.
\DescribeMacro{\pythonimmediateverb}
\DescribeMacro{\pythonimmediate:n}
\DescribeEnv{pythonimmediateenv}
Each of them sends the code to the Python side, then listen for \emph{one} \TeX\ command and execute it
using \cs{__read_do_one_command:}.


\right{execute Python code: print(1)}
\left{execute TeX code: 1}


\right{execute Python code: print(var(a)*2)}
\left{execute TeX code: sendtopy(a); execute another command}
\right{123}
\left{execute TeX code: 123123}


\right{execute Python code: tex.exec(a=456); print(var(a)*2)}
\left{execute TeX code: a=456; sendtopy(done); execute another command}
\right{done}
\left{execute TeX code: sendtopy(a); execute another command}
\right{456}
\left{456456}


An example where the commands are nested follows. It shows that the Python side must not just listen for "done"
command back, but must potentially call a nested loop.

The exact protocol is:
\begin{itemize}
	\item "execute Python code" sends from \TeX\ to Python has the format:
\begin{verbatim}
i
multiple lines of
Python code
here...
pythonimm?"""?'''?
\end{verbatim}

		This can be done by \cs{pythonimmediate:n} or other user interface as listed above.

	\item "done" sends from \TeX\ to Python has the format "|r|\meta{optional return value as a string in a single line}".

		This is sent by executing \TeX\ command |\pythonimmediatecontinue|, which takes a single argument to be e-expanded using |\write|
		as the "return value".

	\item "execute \TeX\ code" sends from Python to \TeX\ must only be sent when the \TeX\ side listens for a command.
		It consist of a single line specify the "command name", which \TeX\ will
		execute the command named
		|\__run_|\meta{command name}|:|
		which must already be defined on the \TeX\ side.

		The command itself might contain additional code to execute more code, e.g. by reading more lines from Python.
\end{itemize}


