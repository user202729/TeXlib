\ProvidesFile{pythonimmediate.tex}[2022/12/21 0.0.0 Library to run Python code]
\RequirePackage{fvextra}
\documentclass{l3doc}
\usepackage{tikz}
\usetikzlibrary{calc}
\usetikzlibrary{arrows.meta}
\EnableCrossrefs
\CodelineIndex
\fvset{breaklines=true,gobble=0,tabsize=4,frame=single,numbers=left,numbersep=3pt}

\AtBeginDocument{\DeleteShortVerb\"}  % https://tex.stackexchange.com/a/650966/250119
\MakeOuterQuote{"}

\newcommand\reg{régime}
\newcommand\DescribePython{\DescribeMacro}  % hack


\begin{document}


% hack to make quotes inside |...| straight
\edef\temp{\def\noexpand|{\noexpand\Verb\string|}}\temp
% basically execute \def| { \Verb ⟨catcode other |⟩ }

\GetFileInfo{\jobname.tex}

\title{\pkg{\jobname} --- \fileinfo
\thanks{This file describes version \fileversion, last revised \filedate.}
}
\author{user202729}
\date{Released \filedate}

\maketitle

\begin{abstract}
	\fileinfo.
\end{abstract}

\section{Motivation}

Just like \pkg{Perl\TeX} or \pkg{PyLua\TeX} (and unlike \pkg{Python\TeX} or \pkg{lt3luabridge}),
this only requires a single run, and variables are persistent throughout the run.

Unlike \pkg{Perl\TeX} or \pkg{PyLua\TeX}, there's no restriction on compiler or script required to run the code.

There's also debugging functionalities -- \TeX\ errors results in Python traceback, and Python error results in \TeX\ traceback.
Errors in code executed with the |pycode| environment gives the correct traceback point to the Python line of code in the \TeX\ file.

For advanced users, this package allows the user to manipulate the \TeX\ state directly from within Python,
so you don't need to write a single line of Python code.

\section{Installation}

%You need to install Python \pkg{daemoniker} and \pkg{psutil} package.

The package should work out of the box with no additional Python package needed.

\subsection{Installation on Overleaf}

At the point of writing, this package can be used on Overleaf.

Instruction:

\begin{itemize}
	\item Download the following files and place it in the root folder of Overleaf:
		\begin{itemize}
			\item \file{saveenv.sty}
			\item \file{precattl.sty}
			\item \file{pythonimmediate.sty}
			\item \file{pythonimmediate_script_textopy.py}
			\item \file{pythonimmediate_script_pytotex.py}
		\end{itemize}
	\item Write the following in the preamble:
\begin{verbatim}
\usepackage[abspath]{currfile}
\usepackage[mode=unnamed-pipe]{pythonimmediate}
\end{verbatim}
Refer to \ref{troubleshoot-source-file-not-found} for explanation of the |abspath| option.
\end{itemize}

For some unknown reason in the default mode on Overleaf (|\nonstopmode|), when there's an error
the log file might be truncated,
so in that case consider writing |\errorstopmode|.

Refer to \ref{troubleshoot-Python-error} to read the error traceback in case of Python error.

\section{Usage}

\subsection{Package options}

\DescribeOption{outputdir=}
Specify the output directory if it's not the default value.

\begin{texnote}
	The value will be |x|-expanded.
\end{texnote}

Using the \pkg{outputdir} package, it's possible to determine the output directory automatically, subject to restrictions.

An usage example, if you invoked |pdflatex| with the command
\begin{verbatim}
pdflatex --output-directory=/tmp/ a.tex
\end{verbatim}
then in the file \file{a.tex} you should have
\begin{verbatim}
\usepackage[outputdir=/tmp/]{pythonimmediate}
\end{verbatim}

\DescribeOption{mode=}
Method to communicate between \TeX\ and Python. Usually the method will be automatically detected.

Possible values include |multiprocessing-network| and |unnamed-pipe|.

\subsection{\TeX\ interface}

The interface mimics those in popular packages such as \pkg{Python\TeX} or \pkg{PyLua\TeX}.

\subsubsection{Inline commands}

\DescribeMacro{\py}
Evaluate some Python expression, consider the result as a string, then execute the result as \TeX\ command.

\begin{texnote}
	The command is not expandable, and the argument will be fully expanded, then detokenized under |\escapechar=-1|, then the resulting
	passed to Python as a string.
\end{texnote}

Which, for the users who are not familiar with \TeX\ terminology, roughly means the following:
\begin{itemize}
	\item the value can only be used to typeset text,
		it must not be used to pass "values" to other \LaTeX\ commands. See the example below.
	\item Special characters can be "escaped" simply by prefixing the character with backslash.
\end{itemize}

Example:
\begin{verbatim}
The value of $1+1$ is $\py{1+1}$.

The following is the \TeX\ logo: \py{'\\\\TeX'}
\end{verbatim}
In the example above, the Python expression being evaluated is |'\\TeX'|, which results in a string that contains the content |\TeX|
which typesets the \TeX\ logo.

The following is illegal, as the result (2) can only be used to typeset text, not passed to another command that expect a "value":
\begin{verbatim}
\setcounter{abc}{\py{1+1}} 
\end{verbatim}
A possible workaround is:
\begin{verbatim}
\py{ r'\\setcounter{abc}{' + str(1+1) + '}' }
\end{verbatim}
In this example you don't need to escape the |{}| characters, but if the Python code has those unbalanced then you can escape them as mentioned above.

\DescribeMacro{\pyc}
Execute some Python code provided as an argument (the argument will be interpreted as described above).

The macro is not expandable -- roughly speaking, you can only use this at "top level".

Any output (as described in \ref{print-to-TeX}) will be typesetted.

The code will be tokenized.

\DescribeMacro{\pycq}
Same as above, but output will not be typesetted.

\DescribeMacro{\pyfile}
Given an argument being the file name, execute that file.

\DescribeMacro{\pys}
Performs "string interpolation", the same way as \pkg{Python\TeX}.
(not yet implemented)

\subsubsection{Environments}

\DescribeEnv{pycode}
Verbatim-like environment that executes the code inside as Python.

Example usage: The following will typeset |123|
\begin{verbatim}
\begin{pycode}
pythonimmediate.print("123")
\end{pycode}
\end{verbatim}

Special note: white spaces at the end of lines are preserved.

Any output (as described in \ref{print-to-TeX}) will be typesetted.

\DescribeEnv{pycodeq}
Same as above, but output will not be typesetted.

\DescribeEnv{pysub}
Not yet implemented.

\subsection{Python interface}

The \TeX\ interface is only used to call Python. Apart from that, all the work can be done on the Python side.

All functions in this section should be imported from |pythonimmediate| package, unless specified otherwise.

\subsubsection{Print to \TeX}\label{print-to-TeX}

\DescribePython{.print()}
\DescribePython{.file}
These functions are used in |\pyc| command or |pycode| environment.

Unlike most other packages, using |print()| function in Python will print to the console (\TeX\ standard output).
In order to print \TeX\ code to be executed, you can do one of
\begin{verbatim}
pythonimmediate.print(...)
print(..., file=pythonimmediate.file)
with contextlib.redirect_stdout(pythonimmediate.file):
	print(...)
\end{verbatim}
Note that in quiet environments, |pythonimmediate.file| is None, the second variant using |print()| will print to stdout
instead of suppress output. The third variant works as expected.

All output will be buffered until the whole Python code finishes executing.
In order to typeset the text immediately use one of the advanced commands.



\DescribePython{.newcommand()}
\DescribePython{.renewcommand()}
Same as \LaTeX's |\newcommand| and |\renewcommand|. Can be used as follows:

\begin{verbatim}
from pythonimmediate import newcommand, renewcommand

@newcommand
def function():
	...
# define |\function| in TeX

@newcommand("controlsequencename")
def function():
	...
# define |\controlsequencename| in TeX

def function():
	...
newcommand("controlsequencename", function)
\end{verbatim}

\DescribePython{.get_arg_str()}
\DescribePython{.get_optional_arg_str()}
\DescribePython{.get_verb_arg()}
\DescribePython{.get_multiline_verb_arg()}
\DescribePython{.peek_next_char()}
\DescribePython{.get_next_char()}
There are those functions that is mostly understandable to an inexperienced \LaTeX\ user,
and should be sufficient for a lot of programming works.

This is an example of how the functions could be used. The name should be mostly self-explanatory.

\begin{verbatim}
\documentclass{article}
\usepackage{pythonimmediate}
\begin{document}
\begin{pycode}
from pythonimmediate import newcommand, peek_next_char, get_next_char, get_arg_str, print
@newcommand
def innerproduct():
	s = get_arg_str()    # in the example below this will have the value '\mathbf{a},\mathbf{b}'
	x, y = s.split(",")  # it's just a Python string, manipulate normally (but be careful of comma inside braces, parse the string yourself)
	print(r"\left\langle" + x + r"\middle|" + y + r"\right\rangle")

@newcommand
def fx():
	if peek_next_char() == "_":
		get_next_char()
		subscript = get_arg_str()
		print("f_{" + subscript + "}(x)")
	else:
		print("f(x)")

@newcommand
def sumManyArgs():
	s = 0
	while peek_next_char() == "{":
		i = get_arg_str()
		s += int(i)
	print(str(s))
\end{pycode}
$1+2+3 = \sumManyArgs{1}{2}{3}$

$\innerproduct{\mathbf{a},\mathbf{b}}=1$

$\fx = 1$, $\fx_i = 2$, $\fx_{ij} = 3$
\end{document}
\end{verbatim}

It will typeset:

\begin{quote}
$1+2+3=6$

$\left\langle\mathbf{a}\middle\vert\mathbf{b}\right\rangle=1$

$f(x)=1$, $f_i(x)=2$, $f_{ij}(x)=3$
\end{quote}


\DescribePython{.execute()}
Takes a string and execute it immediately. (so that any |.execute()| will be executed before any |.print()|)

Assuming \TeX\ is in errorstopmode (i.e. errors halt \TeX\ execution),
any error in \TeX\ will create an error in Python and the traceback should point to the correct line of code.

For example, in the following code

\begin{verbatim}
\documentclass{article}
\usepackage{tikz}
\usepackage{pythonimmediate}
\begin{document}

\begin{tikzpicture}
\begin{pycode}
from pythonimmediate import execute
execute(r'\draw (0, 0) to (1, 1);')
execute(r'\draw (2, 2) to (p);')
execute(r'\draw (3, 3) to (4, 4);')
\end{pycode}
\end{tikzpicture}

\end{document}
\end{verbatim}
each |\draw| command will be executed immediately when the Python |.execute()| function is executed,
and as the second line throws an error, the Python traceback will point to that line.

\section{Troubleshooting}

\subsection{"Source file not found!" error message}\label{troubleshoot-source-file-not-found}

In order to obtain the exact code with trailing spaces and produce error traceback
point to the correct \TeX\ file, the Python code need to know the full path to the current
\TeX\ file for the |pycode| environment.

Nevertheless, this is difficult and does not always work
(refer to the documentation of \pkg{currfile} for details), so this message is issued
when the file cannot be found.

In that case try the following fixes:

\begin{itemize}
	\item Include |\usepackage[abspath]{currfile}| at the start of the document, after the |\documentclass| line.
		(this option is not included by default because it's easy to get package clash, and usually \pkg{currfile} without
		the |abspath| option works fine -- unless custom |jobname| is used)
	\item Explicitly override |currfilename| or |currfileabspath| -- for example
\begin{verbatim}
\def\currfilename{main.tex}
\end{verbatim}
		Technically this is an abuse of the \pkg{currfile} package API, but it usually works regardless.
\end{itemize}

\subsection{"Python error" error message}\label{troubleshoot-Python-error}

In case of Python error, the Python traceback is included in the terminal and \TeX\ log file.

Search for "Python error traceback" before the error line in the log file.

On Overleaf, you can either view the log file ("Raw logs" section)
or the traceback on stderr (download \file{output.stderr} file)

\subsection{"\TeX\ error" error message}\label{troubleshoot-TeX-error}

If an error occur in \TeX, traceback cannot be included in the log file.

Besides, this can only be detected in |\errorstopmode|. Such an error will always halt \TeX,
and Python will be force-exited after printing the error traceback.

On Overleaf, download \file{output.stderr} file to read the traceback.

\section{Implementation note}

Communication between \TeX\ and Python are done by opening two pseudo-files from the output of a Python process |textopy|
(similar to |\ior_shell_open:Nn|)
and to the input of another Python process |pytotex| (this would be |\iow_shell_open:Nn|, if \LaTeX3 have such a function).

There are various methods for the 2 Python child processes to communicate with each other.
After some initial bootstrapping to setup the communication, we can consider only the |textopy| script, the other
merely serves as the bridge to send input to \TeX.

The communication protocol is a little complicated, since it must support nesting bidirectional execution of \TeX\ and Python.

Besides, I believe it's not possible to make a "background listener" on the \TeX\ side, so it must keep track of whether a command should be read from Python and executed.

We have the following 3 internal macros/environments to execute Python code at the lowest level.
\DescribeMacro{\pythonimmediateverb}
\DescribeMacro{\pythonimmediate:n}
\DescribeEnv{pythonimmediateenv}
Each of them sends the code to the Python side, then listen for \emph{one} \TeX\ command and execute it
using |\__read_do_one_command:|.

\ExplSyntaxOn
\let\fpEval\fp_eval:n
\ExplSyntaxOff

\begingroup

\tikzset{myarrow/.style={-{Latex[length=2mm]}}}

\def\blockWidth{1}
\def\blockHeight{7}
\def\separation{10.5}
\def\step{0.7}

\def\left #1{
	\draw [myarrow] (p) ++(\separation, 0) -- node [above] {\small #1} ++(-\separation, 0);
	\path (p) ++ (0, -\step) coordinate (p);
}
\def\right #1{
	\draw [myarrow] (p) -- node [above] {\small #1} ++(\separation, 0);
	\path (p) ++ (0, -\step) coordinate (p);
}

\def\drawBlocks{
	\draw (0, 0) rectangle (\blockWidth, \blockHeight);
	\draw (\fpEval{\blockWidth+\separation}, 0) rectangle (\fpEval{\separation+2*\blockWidth}, \blockHeight);
	\path (\blockWidth, \fpEval{\blockHeight-\step}) coordinate (p);
}

\vspace{5pt}
\begin{tikzpicture}
\edef\blockHeight{\fpEval{\step*3}}
\drawBlocks
\right{execute Python code: print(1)}
\left{execute \TeX\ code: 1}
\end{tikzpicture}

Nevertheless, there may be mored complicated cases where the Python code itself may call \TeX\ code before actually returns:

\vspace{5pt}
\begin{tikzpicture}
\edef\blockHeight{\fpEval{\step*5}}
\drawBlocks
\right{execute Python code: print(var(a)*2)}
\left{execute \TeX\ code: sendtopy(a); execute another command}
\right{123}
\left{execute \TeX\ code: 123123}
\end{tikzpicture}

Or:

\vspace{5pt}
\begin{tikzpicture}
\edef\blockHeight{\fpEval{\step*7}}
\drawBlocks
\right{execute Python code: tex.exec(a=456); print(var(a)*2)}
\left{execute \TeX\ code: a=456; sendtopy(done); execute another command}
\right{done}
\left{execute \TeX\ code: sendtopy(a); execute another command}
\right{456}
\left{456456}
\end{tikzpicture}


An example where the commands are nested follows. It shows that the Python side must not just listen for "done"
command back, but must potentially call a nested loop.

The exact protocol is:
\begin{itemize}
	\item "execute Python code" sends from \TeX\ to Python has a single line "|i|\meta{handler name}",
		followed by any number of arguments (depends on the handler).

		Refer to the |define_TeX_call_Python| internal function for details.

	\item "done" sends from \TeX\ to Python has the format "|r|\meta{optional return value as a string in a single line}".

		This is sent by executing \TeX\ command |\pythonimmediatecontinue|, which takes a single argument to be e-expanded using |\write|
		as the "return value".

	\item "execute \TeX\ code" sends from Python to \TeX\ must only be sent when the \TeX\ side listens for a command.
		It consist of a single line specify the "command name", which \TeX\ will
		execute the command named
		|\__run_|\meta{command name}|:|
		which must already be defined on the \TeX\ side.

		The command itself might contain additional code to execute more code, e.g. by reading more lines from Python.

		Refer to the |define_Python_call_TeX| internal function for details.
\end{itemize}



\PrintIndex

\end{document}
