\ProvidesFile{miniprep.tex}[2022/08/03 0.0.0 Helper to prepare .sty file, alternative to docstrip program]
\documentclass{l3doca}
\begin{document}
\GetFileInfo{\jobname.tex}
\title{\pkg{\jobname} --- \fileinfo
\thanks{This file describes version \fileversion, last revised \filedate.}
}
\author{user202729}
\date{Released \filedate}

\maketitle

\begin{abstract}
Helper to prepare .sty file. Alternative to docstrip program.
\end{abstract}

\section{Usage}

For example, you have a sty file named \file{mypackage.sty}.

Rename it to \file{mypackage-src.sty}, then
add the following 2 lines at the start of the file to use this package's functionality:

\begin{verbatim}
\ProvidesFile{miniprep.tex}[2022/08/03 0.0.0 Helper to prepare .sty file, alternative to docstrip program]
\documentclass{l3doca}
\begin{document}
\GetFileInfo{\jobname.tex}
\title{\pkg{\jobname} --- \fileinfo
\thanks{This file describes version \fileversion, last revised \filedate.}
}
\author{user202729}
\date{Released \filedate}

\maketitle

\begin{abstract}
Helper to prepare .sty file. Alternative to docstrip program.
\end{abstract}

\section{Usage}

For example, you have a sty file named \file{mypackage.sty}.

Rename it to \file{mypackage-src.sty}, then
add the following 2 lines at the start of the file to use this package's functionality:

\begin{verbatim}
\ProvidesFile{miniprep.tex}[2022/08/03 0.0.0 Helper to prepare .sty file, alternative to docstrip program]
\documentclass{l3doca}
\begin{document}
\GetFileInfo{\jobname.tex}
\title{\pkg{\jobname} --- \fileinfo
\thanks{This file describes version \fileversion, last revised \filedate.}
}
\author{user202729}
\date{Released \filedate}

\maketitle

\begin{abstract}
Helper to prepare .sty file. Alternative to docstrip program.
\end{abstract}

\section{Usage}

For example, you have a sty file named \file{mypackage.sty}.

Rename it to \file{mypackage-src.sty}, then
add the following 2 lines at the start of the file to use this package's functionality:

\begin{verbatim}
\ProvidesFile{miniprep.tex}[2022/08/03 0.0.0 Helper to prepare .sty file, alternative to docstrip program]
\documentclass{l3doca}
\begin{document}
\GetFileInfo{\jobname.tex}
\title{\pkg{\jobname} --- \fileinfo
\thanks{This file describes version \fileversion, last revised \filedate.}
}
\author{user202729}
\date{Released \filedate}

\maketitle

\begin{abstract}
Helper to prepare .sty file. Alternative to docstrip program.
\end{abstract}

\section{Usage}

For example, you have a sty file named \file{mypackage.sty}.

Rename it to \file{mypackage-src.sty}, then
add the following 2 lines at the start of the file to use this package's functionality:

\begin{verbatim}
\input{miniprep.tex}
\miniprep{mypackage}
\end{verbatim}

Note that the |.sty| extension should not be included in the parameter.

Optionally define |\miniprepoutputdir{/tmp/}| to specify a particular output directory.
Required if the output directory is not the current directory and shell-escape mode is used (i.e. the compiler is not \LuaLaTeX).
Remember to include the training slash, and don't try to put weird characters in the path.

\section{Why not \pkg{docstrip}?}

I think the main reason is that it's not easy to extend \pkg{docstrip} or \pkg{l3docstrip}
to do what I want here. (preprocess the source code with some custom processor)

\section{Why not \pkg{l3build}?}

Not sure, actually.

I might be reinventing some part of the code.

Although \pkg{l3build} is more focused on the building part instead of the "preprocess the code" part.

\section{Functionality}

Without any change, when the package is run, it will compile \file{mypackage-src.sty}
to \file{mypackage.sty}, then |\input| the resulting file, and execute |\endinput| on the current file.

There are some special directive that invokes specific functionality.



\PrintIndex

\end{document}

\miniprep{mypackage}
\end{verbatim}

Note that the |.sty| extension should not be included in the parameter.

Optionally define |\miniprepoutputdir{/tmp/}| to specify a particular output directory.
Required if the output directory is not the current directory and shell-escape mode is used (i.e. the compiler is not \LuaLaTeX).
Remember to include the training slash, and don't try to put weird characters in the path.

\section{Why not \pkg{docstrip}?}

I think the main reason is that it's not easy to extend \pkg{docstrip} or \pkg{l3docstrip}
to do what I want here. (preprocess the source code with some custom processor)

\section{Why not \pkg{l3build}?}

Not sure, actually.

I might be reinventing some part of the code.

Although \pkg{l3build} is more focused on the building part instead of the "preprocess the code" part.

\section{Functionality}

Without any change, when the package is run, it will compile \file{mypackage-src.sty}
to \file{mypackage.sty}, then |\input| the resulting file, and execute |\endinput| on the current file.

There are some special directive that invokes specific functionality.



\PrintIndex

\end{document}

\miniprep{mypackage}
\end{verbatim}

Note that the |.sty| extension should not be included in the parameter.

Optionally define |\miniprepoutputdir{/tmp/}| to specify a particular output directory.
Required if the output directory is not the current directory and shell-escape mode is used (i.e. the compiler is not \LuaLaTeX).
Remember to include the training slash, and don't try to put weird characters in the path.

\section{Why not \pkg{docstrip}?}

I think the main reason is that it's not easy to extend \pkg{docstrip} or \pkg{l3docstrip}
to do what I want here. (preprocess the source code with some custom processor)

\section{Why not \pkg{l3build}?}

Not sure, actually.

I might be reinventing some part of the code.

Although \pkg{l3build} is more focused on the building part instead of the "preprocess the code" part.

\section{Functionality}

Without any change, when the package is run, it will compile \file{mypackage-src.sty}
to \file{mypackage.sty}, then |\input| the resulting file, and execute |\endinput| on the current file.

There are some special directive that invokes specific functionality.



\PrintIndex

\end{document}

\miniprep{mypackage}
\end{verbatim}

Note that the |.sty| extension should not be included in the parameter.

Optionally define |\miniprepoutputdir{/tmp/}| to specify a particular output directory.
Required if the output directory is not the current directory and shell-escape mode is used (i.e. the compiler is not \LuaLaTeX).
Remember to include the training slash, and don't try to put weird characters in the path.

\section{Why not \pkg{docstrip}?}

I think the main reason is that it's not easy to extend \pkg{docstrip} or \pkg{l3docstrip}
to do what I want here. (preprocess the source code with some custom processor)

\section{Why not \pkg{l3build}?}

Not sure, actually.

I might be reinventing some part of the code.

Although \pkg{l3build} is more focused on the building part instead of the "preprocess the code" part.

\section{Functionality}

Without any change, when the package is run, it will compile \file{mypackage-src.sty}
to \file{mypackage.sty}, then |\input| the resulting file, and execute |\endinput| on the current file.

There are some special directive that invokes specific functionality.



\PrintIndex

\end{document}
