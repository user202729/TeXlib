\ProvidesFile{outputdir.tex}[2022/08/08 0.0.0 Get the current output directory]
\documentclass{l3doca}
\begin{document}
\GetFileInfo{\jobname.tex}
\title{\pkg{\jobname} --- \fileinfo
\thanks{This file describes version \fileversion, last revised \filedate.}
}
\author{user202729}
\date{Released \filedate}
\maketitle

\begin{abstract}
Get the current output directory.
\end{abstract}

\section{Motivation}

Unfortunately, there's no straightforward way to get the output directory in \TeX.

Thus, packages that uses either shell escape or Lua code to write to files
usually require the user to explicitly provide the output directory, or
fails to work if the output directory is not the current working directory
(e.g. \pkg{minted}, externalize feature in \pkg{tikz}).

This package uses various trick to try to determine the output directory.

\section{Usage}

Simply load the package:

\begin{verbatim}
\usepackage{outputdir}
\end{verbatim}

\begin{variable}{\outputdir}
   If the package successfully determine the current output directory, \cs{outputdir} will be defined
   (as an \pkg{expl3} string). Otherwise, it's left undefined.

   Note that this might not have a trailing slash, have one trailing slash, or have two trailing slashes.

   And this might be an absolute path, or relative to the current folder.
\end{variable}

\section{Remark}

The package may create a temporary file named \file{outputdir-tempfile.tex} or similar in its working. Do not
put a similar name for your file.

This package is not guaranteed to work in all cases: it needs either \LuaTeX\ or |-recorder| flag enabled.

There are some corner cases where the |-recorder| flag is initially enabled, then afterwards disabled. In that case the package might
give the wrong result.

If unrestricted |-shell-escape| is enabled, the output directory value is double-checked if |-recorder| is used.
In other words, it will never give wrong result, but might fail to determine the output directory if |-recorder| is disabled.

The limitation of |-recorder| method is the same as \pkg{currfile-abspath} package. As mentioned in that documentation,
on Mik\TeX\ the output directory value is only available since the second run.

\begin{function}{\outputdirstopifundefined}
	If you execute this in the \TeX\ file such as
	\begin{verbatim}
	\usepackage{outputdir}
	\outputdirstopifundefined
	\end{verbatim}
	then the \TeX\ run will stop immediately if \cs{outputdir} cannot be determined.

	(which might be a good idea since it speeds up the run, and gets to the second run more quickly in the Mik\TeX\ case)
\end{function}

\begin{variable}{default}  % TODO hack this is not a macro, this is an option (l3doc does not support this)
Alternatively, you can provide a default value for \cs{outputdir} which will be used (only) if a value cannot be determined:
\begin{verbatim}
\usepackage[default=.]{outputdir}
\end{verbatim}
\end{variable}

\PrintIndex
\end{document}
