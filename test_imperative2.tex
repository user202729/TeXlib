%! TEX program = lualatex
\documentclass[12pt]{article}
\usepackage[paper=a4paper,margin=0.6cm]{geometry}
\usepackage{luacode}
\usepackage{prettytok}
\ExplSyntaxOn \prettyinit: \ExplSyntaxOff
\usepackage{precattl}
\usepackage{genvar}
\usepackage{imperative}
\usepackage{minitest}
\AutoGenerateVariants
\errorcontextlines=100
\begin{document}

\ExplSyntaxOn
\makeatletter



\zdef \ifescapenull {
	\return  { \texconditional { \ifnum 0 > \escapechar } }
}

\begin{addlinemarkerenv}
\precattl_exec:n {

% initially there's some token in front of the input stream, either some character cat other, or blank space cat 10
% that token will be removed
\rdef\remove_next_stringified_character {
	\putnext{\meaning} \expandonce
	\matchrm{#\firstchar}
	\texconditional{\if b #\firstchar}{
		\matchrm{\cStr{lank~space~}~}
	}{
		\matchrm{\cStr{he~character}~#1}
	}
}

}


\precattl_exec:n{



% function that does the following:
% assume initially the input stream has <some tokens> ...... <some token whose \string contains 'A'> ...
% it will replace each character in the previous tokens with a dot (.) until the A is seen
% the time complexity of this algorithm is linear instead of the usual quadratic (at the cost of linear \romannumeral recursion depth)
% see examples/tests below
% note that the newly-put A is letter instead of other catcode
\rdef \stringify_until_A {
	\putnext{\string} \expandonce

	% the following token might either be some character or blank space, need to be careful
	\putnext{\meaning} \expandonce
	\matchrm{#\firstchar}
	\texconditional{\ifx #\firstchar \cO{b}} {
		\matchrm{\cStr{lank~space~}~}  % TODO optimizer should be able to merge two matchrm into one
	}{
		\matchrm{\cStr{he~character}~#1}
		\texconditional{\ifx #1 \cO{A}}{
			\putnext{A}
			\return{}
		}{}
	}
	\putnext{ \expandafter . \romannumeral \stringify_until_A }
	\expandonce
}



% this one does the following thing:
% following in the input stream is ....\relax  with X dots in total (the dot can be any N-type token whose meaning is not \relax)
% #\token is some N-type token
% when called once, it will replace the following in the token list with #\token #\token #\token #\token \relax
% the time complexity of this algorithm is linear instead of the usual quadratic (at the cost of linear \romannumeral recursion depth)
\rdef \replicate_helper #\token {
	\assertisNtype #\token
	\matchrm{#\nexttoken}
	\assertisNtype #\nexttoken

	\texconditional{\ifx #\nexttoken \relax} {
		% done, just return
		\putnext {\relax}
	}{
		% in this case, following in the input stream is ...\relax with X-1 dots left, call recursively
		\rcall {\replicate_helper #\token}
		% then put one token
		\putnext{#\token}
	}
}


% #\repeat: {...}
% #\token: some N-type token
% return \token repeated \repeat times.
% the time complexity of this algorithm is linear instead of the usual quadratic (at the cost of linear \romannumeral recursion depth)
\rdef \replicate #\repeat #\token {
	\putnext {#\repeat \relax}
	\rcall {\replicate_helper #\token}
	\matchrm {#\result \relax}  % magically this still works even if #\token = \relax, because it just remove one relax, everything checks out (although it's cheating)
	\return{#\result}
}

% TODO the code below ought to work but it doesn't. ↑ Ugly

	%\prettyw
	%\pretty{repeat=#\repeat token=#\token}
	%\assignoperate #\result {#\repeat \relax} {
	%	\pretty{token=#\token}
	%	\prettyw
	%	\rcall {\replicate_helper #\token}
	%}
	%\return{#\result}


% usage: \romannumeral \stringify_remove_next_repeated \stringify_remove_next_repeated \stringify_remove_next_repeated \exp_end: ⟨some tokens⟩
% will expand once to ⟨some tokens⟩, but with \string repeatedly applied on the first token and that token removed
% the time complexity of this algorithm is linear instead of the usual quadratic (at the cost of linear \romannumeral recursion depth)
% see test below
\rdef \stringify_remove_next_repeated {
	\rcall{}  % call the following, which is either \stringify_remove_next_repeated (which will stringify-remove next N-1 characters), or \exp_end: (which does nothing)
	% then remove one remaining character
	\putnext{\string} \expandonce
	\rcall{\remove_next_stringified_character}
}


% usage: \stringify_remove_next_repeated_count {<X tokens>}
% stringify-remove that many following tokens
\rdef \stringify_remove_next_repeated_count #\count {
	\assignr #\stringify_remove_next_repeated_seq {\replicate {#\count} \stringify_remove_next_repeated}
	\putnext {\romannumeral #\stringify_remove_next_repeated_seq \exp_end:}
	\expandonce
}


% same as above, but does not stringify or check for space, just remove the next item
\rdef \remove_next_repeated {
	\rcall{}  % call the following, same as above
	% then remove one remaining item
	\matchrm{#1}
}

% similar to \stringify_remove_next_repeated_count but does not do the stringify part
\rdef \remove_next_repeated_count #\count {
	\assignr #\remove_next_repeated_seq {\replicate {#\count} \remove_next_repeated}
	\putnext {\romannumeral #\remove_next_repeated_seq \exp_end:}
	\expandonce
}

% usage: r-expansion of \exp_forward \exp_forward \exp_forward \exp_forward ⟨token list⟩
% is equal to \exp_forward \exp_forward \exp_forward \exp_forward + r-expansion of ⟨token list⟩
\rdef \exp_forward {
	\rcall {}
	\return {\exp_forward}
}

% usage: r-expansion of \exp_forward \exp_forward \exp_forward \exp_forward \exp_forward_end ⟨token list⟩
% is equal to \exp_forward \exp_forward \exp_forward \exp_forward \exp_forward_end + o-expansion of ⟨token list⟩
\rdef \exp_forward_end {
	\expandonce
	\return {\exp_forward_end}
}


\rdef \r_function_do_nothing {}

\rdef \r_function_meaning_noexpand {
	\putnext {\noexpand} \expandonce
	\putnext {\meaning} \expandonce
}

\rdef \r_function_meaning {
	\putnext {\meaning} \expandonce
}

% this function must be followed with a N-type.
% It will \meaning that token, and add a token `1` if and only if the nexttoken is outer.
% See tests below to understand more clearly.
\rdef \check_outer_aux {
	\putnext {\noexpand} \expandonce
	\matchrm {#\dangeroustoken}
	\putnext {\expandafter { \meaning #\dangeroustoken }}
	\expandonce
	\matchrm {#\the_meaning}
	\putnext {#\the_meaning}

	% okay it's easy now
	\assignoperate #\tmp {#\the_meaning \cO{macro:}\relax} {
		\matchrm {#1\cO{macro:}#2\relax}
		\putnext {{#1} {#2}}
	}
	\putnext {#\tmp} \matchrm {#1 #2}


	\conditional{\ifempty_simple{#2}}{
		% case 1, "macro:" string does not appear in its definition, then #2 is empty, it must not be \outer
	}{
		% case 2, "macro:" string does appear in its definition
		% problem: according to https://tex.stackexchange.com/questions/10090/every-possible-meaning-that-a-token-can-have it might be a something-mark
		% check if the string "mark" appear in #1
		% for a real macro it can only be \protected\long\outer, where \ can be any character (escapechar). Either way `mark` will not appear
		\assignoperate #\tmp {#1 \cO{mark}} { \matchrm {#_ \cO{mark}} }
		\conditional{\ifempty_simple{#\tmp}}{
			% good, "mark" does not appear in #1
			% now check if "outer" appear in #1. If it does this is definitely an outer token
			\assignoperate #\tmp {#1 \cO{outer}} {
				\matchrm {#_ \cO{outer}}
			}
			\conditional{\ifempty_simple{#\tmp}}{
			}{
				% it is outer
				\return{1}
			}
		}{}
	}
}

% count the length of the concatenated string representation of all tokens in the o-expansion of #container
%   after #\initial_removable first characters are stringify-removed,
%   then #\preprocess (a r-function) is applied,
%   then #\known_removable next characters are stringify-removed.

% this is pretty complex, see tests below to understand more clearly.

% the time complexity of this one is quadratic, I think (although it could be linear if there's only one A appear in the token list)

\rdef \count_string_len #\container #\initial_removable #\preprocess #\known_removable {
	\assertisNtype #\container
	\assertisNtype #\preprocess
	\assertisExpforwardchain #\known_removable
	\assertisExpforwardchain #\initial_removable

	\putnext {#\container A}
	\expandonce
	% input stream → ⟨dangerous tokens⟩ A

	\rcall {\stringify_remove_next_repeated_count {#\initial_removable}}
	\rcall {#\preprocess}
	\rcall {\stringify_remove_next_repeated_count {#\known_removable}}
	% input stream → ⟨dangerous tokens after first X tokens removed, might be empty⟩ A


	\rcall {\stringify_until_A}
	% now the following in the input stream is either `.........A` or `.........A ⟨remaining potentially dangerous tokens⟩ A`

	\matchrm{#\new_removable A}

	% now, more is known to be removable.
	\assignr #\new_removable {\replicate {#\new_removable} \exp_forward}
	\assertisExpforwardchain #\new_removable

	% then, put ⟨the tokens⟩ B forward in the token list
	\putnext {#\container B}
	\expandonce

	% → ⟨potentially dangerous tokens⟩ B ⟨remaining potentially dangerous tokens⟩ A

	\rcall {\stringify_remove_next_repeated_count {#\initial_removable}}
	\rcall {#\preprocess}
	\rcall {\stringify_remove_next_repeated_count {#\known_removable #\new_removable}}

	\putnext {\string} \expandonce
	% input stream becomes →
	%   A ⟨potentially dangerous tokens⟩ B ⟨potentially dangerous tokens⟩ A
	%   or
	%   B

	\matchrm{#1}  % A or B
	\texconditional{\ifx #1 \cO{A}} {
		% not good, need to keep working
		\assignr #\remaining_len {\count_string_len #\container {#\initial_removable} #\preprocess {#\known_removable #\new_removable \exp_forward}}
		\assertisExpforwardchain #\remaining_len

		% now we know #\remaining_len, and following in the input stream is
		% ⟨remaining_len characters after stringified⟩ B ⟨remaining_len characters after stringified⟩ A
		% need to remove #\remaining_len × 2 + 2 characters
		\rcall {\stringify_remove_next_repeated_count {#\remaining_len #\remaining_len \exp_forward \exp_forward}}
		\return {#\new_removable \exp_forward #\remaining_len}
	} {
		% okay, this is the result
		\return {#\new_removable}
	}
}

% there should be no \relax and no TeX primitive inside #\tokens
\zdef \ifempty_simple #\tokens {\return{
	\ifx \relax #\tokens \relax
		\expandafter \use_i:nn  % blank, return true
	\else
		\expandafter \use_ii:nn  % not blank, return false
	\fi
}}

% the reverse of the above, for a bit more efficiency
\zdef \ifnotempty_simple #\tokens {\return{
	\ifx \relax #\tokens \relax
		\expandafter \use_ii:nn  % blank, return false
	\else
		\expandafter \use_i:nn  % not blank, return true
	\fi
}}

% check if #\a and #\b has same number of tokens
% linear time (the best you can hope given this input format...)
% only work with some kinds of tokens (e.g. \exp_forward which is the one being used in all the code below anyway)
\zdef \iflenequal_simple #\a #\b {
	\putnext {#\a ? #\b +}
	\rcall{\remove_next_repeated_count {#\b}}  % we can see that this will never touch the +
	\matchrm{ #\nexttoken #\skip +}  % here #\nexttoken is ? if and only if the two token lists are equal in length. Remember to clean up the input stream
	\return {\texconditional {\ifx #\nexttoken ?}}
}

\zdef \if_outer #\container {
	\assertisNtype #\container
	\assignr #\string_len { \count_string_len #\container {} \r_function_do_nothing {} }  % ← this is a \exp_forward chain

	% loop counter = string_len-1 down to 0 here
	\assign #\counter {#\string_len}
	\while
	{
		\assignoperate #\counter {#\counter} {\matchrm{#1}}  % decrease by 1

		\assignr #\parta {\count_string_len #\container {#\counter} \r_function_meaning_noexpand {}}
		\assignr #\partb {\count_string_len #\container {#\counter} \r_function_meaning {}}
		\conditional {\iflenequal_simple {#\parta} {#\partb}} {
		}
		{
			% only in this case it might be a outer token (and it definitely is N-type)
			\assignr #\parta {\count_string_len #\container {#\counter} \check_outer_aux {}}
			\conditional {\iflenequal_simple {#\parta} {#\partb}} {
			}{
				% if they're not equal, a `0` token is inserted which means it's outer
				\return {\use_i:nn}
			}
		}
	}{
		\ifnotempty_simple {#\counter}
	}{}

	\return {\use_ii:nn}
}

}
\end{addlinemarkerenv}

\makeatletter
\rdeflinenumbered \debug {} !
\matchrm {12}
\matchrm {34}
!

\directlua{optimize_pending_definitions()}
\directlua{debug_rdef()}
\directlua{print_tlrepr()}
\directlua{execute_pending_definitions()}

%\pretty:o {\romannumeral \debug 12~34 end}




% test for \stringify_until_A
\precattl_exec:n{
\assertequal:on {\romannumeral \stringify_until_A abcAdef} {...Adef}
\assertequal:on {\romannumeral \stringify_until_A abc\xxAyy def} {......A\cO{yy}def}  % inside string representation of other token
\assertequal:on {\romannumeral \stringify_until_A \cS{\ \ \ }Adef} {...Adef}  % can handle space
\begingroup
\escapechar=-1~
\assertequal:on {\romannumeral \stringify_until_A abc\xxAyy def} {.....A\cO{yy}def}  % one shorter if escapechar empty
\endgroup
\assertequal:on {\romannumeral \stringify_until_A abc{}dAdef} {......Adef}  % can break through groups  (a bit too hard to test unbalanced token list...)

% test for \stringify_remove_next_repeated
\assertequal:on {\romannumeral \stringify_remove_next_repeated \stringify_remove_next_repeated \stringify_remove_next_repeated \exp_end: 1~2456} {456}
\assertequal:on {\romannumeral \stringify_remove_next_repeated \stringify_remove_next_repeated \stringify_remove_next_repeated \exp_end: \abcdef ghi} {\cO{cdef} ghi}

% test for \stringify_remove_next_repeated_count
\assertequal:on {\romannumeral \stringify_remove_next_repeated_count {...} \abcdef ghi} {\cO{cdef} ghi}
\assertequal:on {\romannumeral \stringify_remove_next_repeated_count {...} 1~2456} {456}
\assertequal:on {\romannumeral \stringify_remove_next_repeated_count {...} 124~56} {~56}

% test for \remove_next_repeated
\assertequal:on {\romannumeral \remove_next_repeated \remove_next_repeated \remove_next_repeated \exp_end: \abcdef ghi} {i}
\assertequal:on {\romannumeral \remove_next_repeated \remove_next_repeated \remove_next_repeated \exp_end: 1~2456} {56}
\assertequal:on {\romannumeral \remove_next_repeated \remove_next_repeated \remove_next_repeated \exp_end: 124~56} {~56}

% test for \remove_next_repeated_count
\assertequal:on {\romannumeral \remove_next_repeated_count {...} \abcdef ghi} {i}
\assertequal:on {\romannumeral \remove_next_repeated_count {...} 1~2456} {56}
\assertequal:on {\romannumeral \remove_next_repeated_count {...} 124~56} {~56}

% test for \replicate
\assertequal:on {\romannumeral \replicate {...}{\relax}} {\relax\relax\relax}
\assertequal:on {\romannumeral \replicate {}{\relax}} {}
\assertequal:on {\romannumeral \replicate {......}{A}} {AAAAAA}

% test for \exp_forward
\begingroup
\def \__a {\__b}
\def \__b {c}
\assertequal:on {\romannumeral \exp_forward \exp_forward \exp_forward \exp_forward_end \__a} {\exp_forward \exp_forward \exp_forward \exp_forward_end \__b}
\endgroup



}

\precattl_exec:n {
% test for \count_string_len
\begingroup
\tl_set:Nn \__a {12 \cS\ \cS\ 34\__outerexample{\cA\x}\outer ABABA \__outerexample B \__outerexample}
% |12  34\__outerexample{x}\outerABABA\__outerexampleB\__outerexample|
\outer\def \__outerexample{}
\outer\def \cA\x{}
\assertequal:xn {\tl_count:o {\romannumeral \count_string_len \__a {} \r_function_do_nothing {}}} {66}
\endgroup
}

\precattl_exec:n {
% weird space
\tl_set:Nn \__a {\cS\a}
\assertequal:xn {\tl_count:o {\romannumeral \count_string_len \__a {} \r_function_do_nothing {}}} {1}

\tl_set:Nn \__a {\cS\ }
\assertequal:xn {\tl_count:o {\romannumeral \count_string_len \__a {} \r_function_do_nothing {}}} {1}

% weird char code for braces
\tl_set:Nn \__a {\cB\a\cE\ }
\assertequal:xn {\tl_count:o {\romannumeral \count_string_len \__a {} \r_function_do_nothing {}}} {2}
}

\precattl_exec:n {
\tl_set:Nn \__a {\cB\ \cE\a}
\assertequal:xn {\tl_count:o {\romannumeral \count_string_len \__a {} \r_function_do_nothing {}}} {2}
}

\precattl_exec:n {
\tl_set:Nn \__a {\cB\ \cE\ }
\assertequal:xn {\tl_count:o {\romannumeral \count_string_len \__a {} \r_function_do_nothing {}}} {2}
}

\precattl_exec:n {
% only return the not known-removable part
\assertequal:xn {\tl_count:o {\romannumeral \count_string_len \__a {} \r_function_do_nothing {\exp_forward}}} {1}
\assertequal:xn {\tl_count:o {\romannumeral \count_string_len \__a {} \r_function_do_nothing {\exp_forward \exp_forward}}} {0}

}



\precattl_exec:n {
	% test for \check_outer_aux
	\begingroup
	\assertequal:on {\romannumeral \check_outer_aux abc} {\cStr{the~letter~a}bc}

	\def\__a {}
	\assertequal:on {\romannumeral \check_outer_aux \__a bc} {\cStr{macro:->}bc}

	\outer\def\__a {}
	\expandafter \assertequal:nn \expandafter {\romannumeral \check_outer_aux \__a bc} {1\cStr{\\outer~macro:->}bc}

	\long\outer\def\__a {}
	\expandafter \assertequal:nn \expandafter {\romannumeral \check_outer_aux \__a bc} {1\cStr{\\long\\outer~macro:->}bc}

	\long\outer\def\cA\x {}
	\expandafter \assertequal:nn \expandafter {\romannumeral \check_outer_aux \cA\x bc} {1\cStr{\\long\\outer~macro:->}bc}
	\endgroup
}


\precattl_exec:n {
	% finally, test for \if_outer
	\begingroup

	\def \__a { \__dangerous }
	\def \__b { \__dangerous \__dangerous }
	\def \__c { \cA\x }
	\def \__d { \cA\x \__dangerous }
	\def \__e { safe { safe \__dangerous safe } safe }
	\outer\def \__dangerous {}
	\outer\def \cA\x {}

	\if_outer \__a {} {\testfail:}
	\if_outer \__b {} {\testfail:}
	\if_outer \__c {} {\testfail:}
	\if_outer \__d {} {\testfail:}
	\if_outer \__e {} {\testfail:}


	\endgroup

	% these does not contain any outer tokens
	\def \__a {\outer\cStr{~macro:->}}
	\if_outer \__a {\testfail:n{1}}{}

	\def \__a {\cStr{\\outer~macro:->}}
	\if_outer \__a {\testfail:n{2}}{}

	\def \__b {\cStr{}}
	\def \__a {\__b}
	\if_outer \__a {\testfail:n{3}}{}

	\def \__b \cStr{\\outer~macro:->} {}  % the "outer macro" string appears in the param text instead of the replacement text
	\def \__a {\__b}
	\if_outer \__a {\testfail:n{4}}{}

	% didn't test with the marks however
}




% great, this can be used to store a sequence of numbers

\def \__a {\__b}
\def \__b {\__c}
\precattl_exec:n {
	\def \numbersplit { \expandafter \cS\  \expandafter \numbersplit \number }
	\def \numberend { \expandafter \cS\  \expandafter \numberend }
}

\pretty:o { \number 123 \numbersplit 456 \numbersplit 789 \numberend \__a }
% →                 123 \numbersplit 456 \numbersplit 789 \numberend \__b

\pretty:o { \expandafter { \number 123 \numbersplit 456 \numbersplit 789 \expandafter } \__a }
% →                      {         123 \numbersplit 456 \numbersplit 789              } \__b }

\prettystop

\end{document}
