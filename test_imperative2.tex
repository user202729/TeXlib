%! TEX program = lualatex
\documentclass[12pt]{article}
\usepackage[paper=a4paper,margin=0.6cm]{geometry}
\usepackage{luacode}
\usepackage{prettytok}
\ExplSyntaxOn \prettyinit: \ExplSyntaxOff
\usepackage{precattl}
\usepackage{genvar}
\usepackage{imperative}
\usepackage{minitest}
\AutoGenerateVariants
\errorcontextlines=100
\begin{document}

\ExplSyntaxOn
\makeatletter



\zdef \ifescapenull {
	\return  { \texconditional { \ifnum 0 > \escapechar } }
}

\addlinemarker {} !
\precattl_exec:n {

% initially there's some token in front of the input stream, either some character cat other, or blank space cat 10
% that token will be removed
\rdef\remove_next_stringified_character {
	\putnext{\meaning} \expandonce
	\matchrm{#\firstchar}
	\texconditional{\if b #\firstchar}{
		\matchrm{\cO{lank}~\cO{space}~} \matchrm{~}
	}{
		\matchrm{\cO{he}~\cO{character}~#1}
	}
}

}!

\use_none:n{{{


% usage: \checkouter { ... }
% must start with a {.
% will eventually execute \pretty{true} or \pretty{false} depends on whether or not there is an outer macro there.
\rdeflinenumbered\checkouter {}!

	% first remove the initial { token
	\putnext{\string} \expandonce  \matchrm{#\first_lbrace_stringified}

	\assignc #\degree{1}
	\assignc #\tokennumber{0}  % an arbitrary number that is incremented every time a token is processed by this algorithm
	\while {
		\assignoo #\degree_as_number {\cs_to_str:N \degree}
	}{
		\texconditional{\ifnum 0=#\degree_as_number~}
	}{
		\putnext{\meaning} \expandonce

		\matchrm{#\firstchar}

		% check if it's "begin-group character"
		\texconditional{\if b #\firstchar}{
			\matchrm{#\secondchar}
			\texconditional{\if e #\secondchar}{
				\matchrm{#\thirdchar}
				\texconditional{\if g #\thirdchar}{
					% okay, is begin-group character
					\assignc #\degree{\the\numexpr\cs_to_str:N#\degree+1}
					\matchrm{in-group~character~}
					\rcall {\remove_next_stringified_character}
				}{
				}
			}{
			}
		}{
			% check if it's "end-group character"
			\matchrm{#\secondchar}
			\texconditional{\if n #\secondchar}{
				\matchrm{#\thirdchar}
				\texconditional{\if d #\thirdchar}{
					% problem: it might be just "\end" (escapechar=-1|32) followed by an outer token.
					% Need to \meaning the next token to tell?

					\putnext{\meaning} \expandonce
					% now the following might be "the character -" or some outer token, or even { / } etc.

					\assignc #\degree{\the\numexpr\cs_to_str:N#\degree-1}
					\matchrm{group~character~}
					\rcall {\remove_next_stringified_character}
				}{
				}
			}{
			}
			
		}



	}

	% finally done, degree=0 now
!


}}}


\makeatletter
\zdeflinenumbered \debug {} !
	\putnextwithexpand {
		\begingroup
		\escapechar=-1~
		*
		\pretty:n { \meaning \noexpand \@@end }
		\endgroup
	} {
		\onlabel * \expandat \noexpand
		\onlabel * \expandat \meaning
	}
!

\addlinemarker {} !
\precattl_exec:n{



% function that does the following:
% assume initially the input stream has <some tokens> ...... <some token whose \string contains 'A'> ...
% it will replace each character in the previous tokens with a dot (.) until the A is seen
% the time complexity of this algorithm is linear instead of the usual quadratic (at the cost of linear \romannumeral recursion depth)
% see examples/tests below
\rdef \stringify_until_A {
	\putnext{\string} \expandonce

	% the following token might either be some character or blank space, need to be careful
	\putnext{\meaning} \expandonce
	\matchrm{#\firstchar}
	\texconditional{\ifx #\firstchar b} {
		\matchrm{lank~space\cS{\ \ }}  % TODO merge two matchrm into one
	}{
		\matchrm{\cO{he}~\cO{character}~#1}
		\texconditional{\ifx #1 \cO{A}}{
			\putnext{A}
			\return{}
		}{}
	}
	\putnext{ \expandafter . \romannumeral \stringify_until_A }
	\expandonce
}



% this one does the following thing:
% following in the input stream is ....\relax  with X dots in total (the dot can be any N-type token whose meaning is not \relax)
% #\token is some N-type token
% when called once, it will replace the following in the token list with #\token #\token #\token #\token \relax
% the time complexity of this algorithm is linear instead of the usual quadratic (at the cost of linear \romannumeral recursion depth)
\rdef \replicate_helper #\token {
	\assertisNtype #\token
	\matchrm{#\nexttoken}
	\assertisNtype #\nexttoken

	\texconditional{\ifx #\nexttoken \relax} {
		% done, just return
		\putnext {\relax}
	}{
		% in this case, following in the input stream is ...\relax with X-1 dots left, call recursively
		\rcall {\replicate_helper #\token}
		% then put one token
		\putnext{#\token}
	}
}


% #\repeat: {...}
% #\token: some N-type token
% return \token repeated \repeat times.
% the time complexity of this algorithm is linear instead of the usual quadratic (at the cost of linear \romannumeral recursion depth)
\rdef \replicate #\repeat #\token {
	\putnext {#\repeat \relax}
	\rcall {\replicate_helper #\token}
	\matchrm {#\result \relax}  % magically this still works even if #\token = \relax, because it just remove one relax, everything checks out (although it's cheating)
	\return{#\result}
}

% TODO the code below ought to work but it doesn't. ↑ Ugly

	%\prettyw
	%\pretty{repeat=#\repeat token=#\token}
	%\assignoperate #\result {#\repeat \relax} {
	%	\pretty{token=#\token}
	%	\prettyw
	%	\rcall {\replicate_helper #\token}
	%}
	%\return{#\result}


% usage: \romannumeral \stringify_remove_next_repeated \stringify_remove_next_repeated \stringify_remove_next_repeated \exp_end: ⟨some tokens⟩
% will expand once to ⟨some tokens⟩, but with \string repeatedly applied on the first token and that token removed
% the time complexity of this algorithm is linear instead of the usual quadratic (at the cost of linear \romannumeral recursion depth)
% see test below
\rdef \stringify_remove_next_repeated {
	\rcall{}  % call the following, which is either \stringify_remove_next_repeated (which will stringify-remove next N-1 characters), or \exp_end: (which does nothing)
	% then remove one remaining character
	\putnext{\string} \expandonce
	\rcall{\remove_next_stringified_character}
}


% same as above, but does not stringify or check for space, just remove the next item
\rdef \remove_next_repeated {
	\rcall{}  % call the following, same as above
	% then remove one remaining item
	\matchrm{#1}
}

% usage: r-expansion of \exp_forward \exp_forward \exp_forward \exp_forward ⟨token list⟩
% is equal to \exp_forward \exp_forward \exp_forward \exp_forward + r-expansion of ⟨token list⟩
\rdef \exp_forward {
	\rcall {}
	\return {\exp_forward}
}

% usage: r-expansion of \exp_forward \exp_forward \exp_forward \exp_forward \exp_forward_end ⟨token list⟩
% is equal to \exp_forward \exp_forward \exp_forward \exp_forward \exp_forward_end + o-expansion of ⟨token list⟩
\rdef \exp_forward_end {
	\expandonce
	\return {\exp_forward_end}
}


% count the length of the concatenated string representation of all tokens in the o-expansion of #container
% the time complexity of this one is quadratic, I think (although it could be linear if there's only one A appear in the token list)

% #\known_removable is a sequence of X \exp_forward tokens, that represents if X \stringify_remove_next_repeated operations are applied on #container, it will remain nonempty

\rdef \count_string_len #\container #\known_removable {
	\assertisNtype #\container

	\putnext {#\container A}
	\rcall {#\known_removable \exp_forward_end} \matchrm{#\known_removable \exp_forward_end}
	% input stream → ⟨dangerous tokens⟩ A

	\assignr #\stringify_remove_next_repeated_seq {\replicate {#\dots} \stringify_remove_next_repeated}
	\rcall {#\known_removable \exp_forward_end} \matchrm{#\known_removable \exp_forward_end}
	% input stream → ⟨dangerous tokens after first X tokens removed, might be empty⟩ A


	% call \stringify_until_A
	\rcall {#\known_removable \exp_forward_end \stringify_until_A} \matchrm{#\known_removable \exp_forward_end}
	% now the following in the input stream is either `.........A` or `.........A ⟨remaining potentially dangerous tokens⟩ A`

	\matchrm{#\dots A}

	% convert each dot to a \exp_forward
	\assignr #\dots {\replicate {#\dots} \exp_forward}

	% then, put ⟨the tokens⟩ B forward in the token list
	\putnext {#\container B}
	\rcall {#\dots \exp_forward_end} \matchrm{#\dots \exp_forward_end}   % each line of this is equivalent to a expandonce, but we can't do that directly because of #\dots


	% → ⟨potentially dangerous tokens⟩ B ⟨remaining potentially dangerous tokens⟩ A


	% now we remove ⟨count⟩ of first following characters using \stringify_remove_next_repeated, then check if the following token is A or B
	\assignr #\stringify_remove_next_repeated_seq {\replicate {#\dots} \stringify_remove_next_repeated}

	\putnext {#\stringify_remove_next_repeated_seq}
	\rcall {#\dots \exp_forward_end} \matchrm{#\dots \exp_forward_end}

	\putnext {\string}
	\rcall {#\dots \exp_forward_end} \matchrm{#\dots \exp_forward_end}

	% input stream becomes →
	%   A ⟨potentially dangerous tokens⟩ B ⟨potentially dangerous tokens⟩ A
	%   or
	%   B

	\matchrm{#1}
	\texconditional{\ifx #1 \cO{A}} {
		% not good, need to keep working
	} {
		% okay, this is the result
	}

	
}

}
!




\directlua{optimize_pending_definitions()}
\directlua{debug_rdef()}
\directlua{debug_rdef2()}
\directlua{execute_pending_definitions()}

%\debug

% test for \stringify_until_A
\precattl_exec:n{
\assertequal:on {\romannumeral \stringify_until_A abcAdef} {...Adef}
\assertequal:on {\romannumeral \stringify_until_A abc\xxAyy def} {......A\cO{yy}def}  % inside string representation of other token
\begingroup
\escapechar=-1~
\assertequal:on {\romannumeral \stringify_until_A abc\xxAyy def} {.....A\cO{yy}def}  % one shorter if escapechar empty
\endgroup
\assertequal:on {\romannumeral \stringify_until_A abc{}dAdef} {......Adef}  % can break through groups  (a bit too hard to test unbalanced token list...)

% test for \stringify_remove_next_repeated
\assertequal:on {\romannumeral \stringify_remove_next_repeated \stringify_remove_next_repeated \stringify_remove_next_repeated \exp_end: 1~2456} {456}
\assertequal:on {\romannumeral \stringify_remove_next_repeated \stringify_remove_next_repeated \stringify_remove_next_repeated \exp_end: \abcdef ghi} {\cO{cdef} ghi}

% test for \remove_next_repeated
\assertequal:on {\romannumeral \remove_next_repeated \remove_next_repeated \remove_next_repeated \exp_end: \abcdef ghi} {i}
\assertequal:on {\romannumeral \remove_next_repeated \remove_next_repeated \remove_next_repeated \exp_end: 1~2456} {56}
\assertequal:on {\romannumeral \remove_next_repeated \remove_next_repeated \remove_next_repeated \exp_end: 124~56} {~56}

% test for \replicate
\assertequal:on {\romannumeral \replicate {...}{\relax}} {\relax\relax\relax}
\assertequal:on {\romannumeral \replicate {}{\relax}} {}
\assertequal:on {\romannumeral \replicate {......}{A}} {AAAAAA}

% test for \exp_forward
\begingroup
\def \__a {\__b}
\def \__b {c}
\assertequal:on {\romannumeral \exp_forward \exp_forward \exp_forward \exp_forward_end \__a} {\exp_forward \exp_forward \exp_forward \exp_forward_end \__b}
\endgroup



}






% great, this can be used to store a sequence of numbers

\def \__a {\__b}
\def \__b {\__c}
\precattl_exec:n {
	\def \numbersplit { \expandafter \cS\  \expandafter \numbersplit \number }
	\def \numberend { \expandafter \cS\  \expandafter \numberend }
}

\pretty:o { \number 123 \numbersplit 456 \numbersplit 789 \numberend \__a }
% →                 123 \numbersplit 456 \numbersplit 789 \numberend \__b

\pretty:o { \expandafter { \number 123 \numbersplit 456 \numbersplit 789 \expandafter } \__a }
% →                      {         123 \numbersplit 456 \numbersplit 789              } \__b }

\prettystop

\end{document}
